\documentclass{amsart}
\usepackage{fullpage}

\newcommand{\hcp}{\textsc{hcp}}
\newcommand{\bal}{\textsc{bal}}
\newcommand{\ak}{\textsc{ak}}
\newcommand{\gf}{\textsc{gf}}
\renewcommand{\c}{\ensuremath{\clubsuit}}
\renewcommand{\d}{\ensuremath{\diamondsuit}}
\newcommand{\h}{\ensuremath{\heartsuit}}
\newcommand{\s}{\ensuremath{\spadesuit}}
\newcommand{\nt}{\textsc{nt}}
\newcommand{\+}{\ensuremath{^+}}
\newcommand{\zoom}{\underline}

\begin{document}

\section*{A two-page guide to Transfer-Oriented Symmetric Relay}

\subsection*{Basic idea}
This system covers the sequences 1\c\ (any 16\+ \hcp) -- 1\h\+ (\gf).
(A \gf\ response requires 2 \ak; see ``Controls.'')  The 1\c\ opener
is now the \emph{relayer} or \emph{asker} and will continue making the
cheapest available bid (with two exceptions, see below) until able to
place the contract.  The other hand is called the \emph{relay
  responder} and makes a sequence of descriptive bids (separated by
semicolons here).  Ideally, the relayer will be the declarer and the
defenders will have no information about the closed hand; hence
``transfer-oriented.''

\subsection*{Very Important Exceptions (signoffs)}
If the relayer bids 3\nt, that is always to play.  A bid of 4\d\ by
the relayer is the \emph{end signal}: responder bids 4\h\ whereupon
the relayer places the contract.  But with a very good hand responder
can bid over 3\nt\ or 4\d\ showing controls, starting with 4\c/4\s\ =
4 \ak.  (See ``Controls.'')  When the next step is 3\nt\ or 4\d,
relayer can continue relays by bidding one step higher instead.

\subsection*{Responder's bids}
The responding hand is described in four stages: shape, range,
controls, and denial cuebidding (location of specific honors).
Sometimes a single bid belongs to two or more stages, thanks to a
technique called \emph{zooming}: when responder would make the highest
possible bid at the end of one stage, responder also begins the next
stage as though the relayer had made a relay ask one step below that
bid.  For example, the sequence 1\s; 3\d\ just shows 3424 shape, but
1\s; 3\h\ shows 4243 shape and a minimum, 1\s; 3\s\ shows 4243 shape
and a maximum with 3 \ak, etc., because there is no shape assigned to
1\s; 3\s.

\subsection*{Shape}
Hands are divided into one-suited, two-suited, three-suited and
balanced (for our purposes defined as 4432 or 4333) shapes.  A
``suit'' is a suit of at least 4 cards, provided the hand is not
balanced.  Every shape that does not have an 8-card suit or 12 cards
in two suits is shown exactly, except that the 2- and 3-card suits of
a 7321-type hand are not distinguished.  With extreme shape, make the
smallest lie possible.

\subsubsection*{One-suiters}
First bid your suit using the scheme 1\h\ = \s, 1\nt\ = \h, 2\c\ = \d,
2\d\ = \c.  (This ensures that the relayer will play the hand if we
play in responder's long suit, provided it is not \h.)  Your next bid
will be 2\s\ or higher.  Look at your length in your other three suits
and see if you have a unique shortest suit.  If so, bid 2\s\ if it is
the highest side suit, bid 2\nt\ if it is the middle side suit, and do
nothing for now (a kind of zoom) if it is the lowest side suit; then
show your shape: 3\d\ = 5332, 3\h\ = 6331, 3\s\ = 7330, \zoom{3\nt} =
7321.  The underline is a reminder to zoom to range and controls,
since 7321 is the highest possible shape on this sequence.  If you do
not have a unique shortest suit, your shape must be 6322 or 7222, and
the sequences are 2\s; 3\c; 3\h = 6223, 2\s; 3\c; \zoom{3\s} = 6232,
3\c; 3\h = 7222, 3\c; \zoom{3\s} = 6322.  (How do you remember these
sequences?  The general rule is increasing numeric order; the general
exception is that when the highest bid would then show a rarer shape
than the bid one lower, we switch them, because zooming saves
\emph{two} steps.)

\subsubsection*{Two-suiters}
Generally begin by bidding both your suits according to the one-suited
scheme.  However, partner's relay asks will prevent you from making
two consecutive bids!  So, there are special sequences for showing
\d+\h\ or \c+\d: with \d+\h\ bid 1\s; 2\c, and with \c+\d\ do nothing
for now; your first bid will be 2\h\ or higher, showing a two-suiter
with the minors.  There's one other slight adjustment: your next bid
might be 2\h, but you can't do that if you bid 2\d\ showing \c.  So
showing a major and then bidding 2\h\ shows \c, also, as though it
were 1\h/1\nt; 2\d; 2\h.  (We didn't use the 2\h\ rebid for
one-suiters, so this is OK.)

Suppose first that one of your suits is only 4 cards long.  Bid
2\h\ now if it's the higher one.  Next bidding 3\c\ shows 5422 and
\zoom{4\c} shows 7411.  Otherwise, one of your short suits is shorter
than the other; bid 2\nt\ if it's the higher one.  Then show your
shape: 3\d\ = 5431, 3\h\ = 6421, 3\s\ = 6430, 3\nt\ = 7420 (zoom from
3\nt\ only if you bid 2\nt\ first).  (Generally when $x$ shows two
equal lengths, $x-1$ shows ``high shortage.'')

If instead both your suits have at least 5 cards, bid 2\s.  Now
3\d\ shows singletons in both short suits, followed by 3\s\ = 5611,
\zoom{3\nt} = 6511.  Otherwise, bid 3\c\ if your higher short suit is
shorter, then 3\h\ = 5521, 3\s\ = 5530, 3\nt\ = 5620, \zoom{4\c} =
6520.  (The first two digits in these numbers are the lengths of your
long suits, with the higher one first; the second two are the lengths
of your short suits with the longer one first.)

\subsubsection*{4432 and 4333 shapes}
With these balanced shapes, start with 1\s\ (so that relayer will
declare a notrump contract).  Your next bid will be 2\d\ or higher.
With 4432, consider your two 4-card suits; bid 2\d\ if they are the
same color, 2\h\ if they are the same rank, and do nothing for now if
they are the same shape; then bid your doubleton: 2\nt\ = \s, 3\c\ =
\c, 3\d\ = \d, \zoom{3\h} = \h.  With 4333, bid 2\d\ with a 4-card
major, then 2\s\ always, then 3\c\ with long \c\ or \h\ and \zoom{3\d}
with long \d\ or \s.

\subsubsection*{Three-suiters}
These are last because they live in the gaps formed by the rest of the
system.  The two unused sequences are 1\h; 1\nt; 2\d\ and 2\c; 2\h.
The first shows a three-suiter with both majors, the second a
three-suiter with both minors.  After one of these starts, there are
only 8 possible shapes, so the follow-ups are simple and step-based:
bid the cheapest step first if you have a high short suit (\s\ or \d),
then bid in steps showing the lengths of your long suits in
\s\h\d\c\ order (444, 445, 454, 544---and zoom with 544).

\subsection*{Range}
Bid the cheapest step with a minimum, otherwise zoom on to controls.
A minimum is a hand with fewer than 12 \hcp\ or fewer than 3 \ak.  A
passed hand is always a minimum and skips this stage.  

\subsection*{Controls}
We define \ak\ as a measure of controls: an Ace is worth 2 \ak, a King
1 \ak.  We don't count any \ak\ for shortness, since relayer already
knows responder's shape.  \ak\ points are shown in steps, starting
with 2 (and zooming with 5) if responder has shown a minimum, 3 (never
zooming) if a maximum.

\subsection*{Denial cuebidding}
Sort your suits by length with the longer suits first, breaking ties
by putting higher ranking suits before lower ones.  You will cycle
through the suits in this order, looking at each suit a number of
times depending on its length: three times for suits of at least 4
cards, twice for 3-card suits, once for 2-card suits, and never for
singletons or voids.  The first time you look at a suit, check whether
it contains exactly one of the Ace or King; the second time, check
whether it contains the Queen; the third time, check whether it
contains the Jack.  (There is an exception for suits of 6 or more
cards described below.)  So, once you have all these ``yes'' or ``no''
answers, what do you \emph{bid}?  If the answer to the first question
is ``no,'' bid the cheapest step.  In general, skip a number of steps
equal to the number of ``yes'' answers.  If relayer continues, start
again where you left off, with the answer after the first ``no.''  In
these auctions, any non-relay bid by the relayer is to play, as is any
bid at the slam level; the highest relay ask is 5\nt.

\subsubsection*{Long suit exception}
When checking a suit of 6 or more cards, replace the first two checks
with the following: first check whether it contains at least two of
the Ace, King and Queen, then check whether it contains an odd number
(the larger possibility given the answer to the first check).

\subsection*{Examples}  In each example North deals and opens 1\c.
\[
\begin{tabular}{ll|ll|lll|ll|l}
1.
&\s\ KQ73&1\c&2\c&\d&                     \qquad 2.&\s\ A9&1\c&1\s&balanced or \d+\h\\
&\h\ AKJ2&2\d&2\h&three-suited with \c+\d&         &\h\ AQ1072&1\nt&2\h&same rank\\
&\d\ Q10&2\s&3\c&4144&                             &\d\ AQJ&2\s&2\nt&2344\\
&\c\ Q107&3\d&3\nt&maximum, 4 \ak&                 &\c\ AK9&3\c&3\d&minimum\\
&&4\d&4\h&end signal&                              &&3\h&3\s&2 \ak\\
&\s\ J654&4\s&---&9 \ak\ is not enough&            &\s\ J2&4\c&4\h&\d K, no \c K\\
&\h\ 4&&&\;\;for slam without&                     &\h\ K98&4\s&5\c&\h K, no \s K\\
&\d\ K954&&&\;\;more distribution.&                &\d\ K642&5\d&5\h&no \d Q\\
&\c\ AKJ6&&&&                                      &\c\ J732&5\s&5\nt&no \c Q\\
\multicolumn{5}{}&&                                &&6\nt&---&7\nt\ is at best\\
\multicolumn{5}{}&&                                &&&&\;\;on a finesse.
\end{tabular}
\]

\subsection*{Further avenues}
We have described a simple, effective system for game-forcing auctions
after a 1\c\ opener.  There is still room for extension and
improvement, though.
\begin{itemize}
\item After a 1\c\ opening and a negative 1\d\ response, 1\h\ can be
  played as any 20\+ \hcp, with 1\s\ a second negative and all the
  relays shifted up two steps.  (There is no range ask, and controls
  are shown starting with 0 \ak.)  In fact, there's an entire system
  in which most of the game-forcing auctions use similar relays after
  limited opening bids as well.
\item We have not assigned meanings to any of relayer's non-relay bids
  below 3\nt\ or to jumps to game in a suit.  The former can be used
  to show minimum 3-suiters (``reverse relay'') or to ask about
  stoppers in side suits; the latter can be played as keycard asks or
  as nonforcing slam tries (``optional RKC'').
\item Some denial cuebidding auctions contain redundant information,
  such as the 5\c\ bid in the second example; South has shown only two
  \ak\ and has already shown the \d K and \h K so cannot have the \s K
  as well.  The 5\c\ bid could therefore deny the \d Q instead, saving
  two bids.  But it's tricky to make sure that both partners make the
  same inferences in all situations.
\end{itemize}

\subsection*{That's it}
Enjoy! \hfill \textsl{--- Reid Barton, September 17, 2008}

\end{document}
