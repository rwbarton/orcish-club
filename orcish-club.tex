\documentclass{report}
\usepackage{fullpage, url}

\newcommand{\fs}{1st/2nd}
\newcommand{\tf}{3rd/4th}
\newcommand{\hcp}{\textsc{hcp}}
\newcommand{\bal}{\textsc{bal}}
\newcommand{\ak}{\textsc{ak}}
\newcommand{\gf}{\textsc{gf}}
\newcommand{\inv}{\textsc{inv}}
\newcommand{\nf}{\textsc{nf}}
\newcommand{\ph}{\textsc{ph}}
\newcommand{\uph}{\textsc{uph}}
\renewcommand{\c}{\ensuremath{\clubsuit}}
\renewcommand{\d}{\ensuremath{\diamondsuit}}
\newcommand{\h}{\ensuremath{\heartsuit}}
\newcommand{\s}{\ensuremath{\spadesuit}}
\newcommand{\nt}{\textsc{nt}}
\newcommand{\p}{\textsc{pass}}
\newcommand{\x}{\textsc{x}}
\newcommand{\y}{\textsc{y}}
\newcommand{\art}{\textsc{art}}
\newcommand{\nat}{\textsc{nat}}
\newcommand{\+}{\ensuremath{^+}}
\newcommand{\M}{\textsc{m}}
\newcommand{\OM}{\textsc{om}}
\newcommand{\modifications}{\paragraph{Modifications from Revision.}}
\newcommand{\zoom}{\underline}
\newcommand{\OR}{\textbf{or}}
\newcommand{\hsmash}[1]{\hbox to 0pt{#1\hss}}

\title{The Orcish Club}
\begin{document}
\maketitle
\tableofcontents
\listoffigures

\chapter{Opening bid structure}

In \fs, all hands with 17\+ \hcp\ or decent 15\+ \hcp\ unbalanced open
1\c, \emph{except} 20--21 \bal\ hands, which open 2\nt.  The exception
is to help us show balanced hands in competition, by passing with a
minimum hand and bidding cheaply in \nt\ with 22\+.  We may rarely
open 1\c\ with 13 or 14 \hcp\ hands of unusual playing strength.  In
\tf, all these ranges are 1 point higher.

We open most 11 \hcp\ and good unbalanced 10 (rarely 9) \hcp\ hands in
\fs.  Hands with a five-card major open 1\h\ or 1\s\ (unless suitable
for 1\nt); hands with a six-card minor, but not five cards in the
other minor, open 2\c\ or 2\d; and the remaining unbalanced hands open
1\d.  Balanced hands open 1\nt\ with 14--16, 1\d\ with 11--13.  We
consider any 4333, 4432, 5332, 5m422, and some 6m322 (with many honors
in the short suits) to be balanced, although 1\d\ may never contain a
five-card major.  We might treat a 4441 or 5431 with a stiff Ace or
King as balanced.

% XXX What do we do with a maximum 5M6m?

In \tf, our opening ranges are nominally all raised by 1 point.  In
3rd however it is rare for responder to have an invite opposite 11--16
without a good fit, so we may feel free to open with 9 or 10 \hcp\ and
a reason to want to bid.  (We are especially likely to stretch a 2m
opener; we would probably never stretch a 1\d\ opener.)  Our
1\nt\ opening is 15--17 in \tf, 2\nt\ 21--22.

\chapter{The 1\c\ opening}

We first describe responses to a \fs\ 1\c\ opening.  Responses to a 
\tf\ 1\c\ opener are similar; in section \ref{tf1c} we describe the changes
when responder is a passed hand.

1\d\ is the negative response, normally 0--8 \hcp, but could contain a
stronger hand with fewer than 2 \ak.  1\h\+ is \gf, transfer-oriented
symmetric relay, promising 2 \ak.

\section{1\c--1\d}
1\h\ shows any 19\+ \hcp\ unbalanced, 21\+ \hcp\ balanced.  1\nt\ shows 16--18 \bal\ (including 5 card majors).  Other rebids are artificial.

\begin{figure}[h]
\begin{tabular}{rll}
Opener's rebid & Description & Artificial inquiry\\
\hline
1\s & At least one long minor, with a 4-card major possible. & 1\nt\\
2\c & Both majors. & 2\d\\
2\d & One major, 6\+. & 2\nt\\
2\M & 5\+\ of the bid major and a 4\+-card minor. & 2\M+1\\
2\nt & 6m322 any quality, good for declaring \nt, almost worth 1\h. & 3\s\\
3\x & Natural, long good suit. & ---\\
3\nt & To play, long running suit with side stoppers. & ---\\
4\M & To play. & ---\\
\end{tabular}
\caption{1\c - 1\d}
\end{figure}

Even the artificial rebids are not forcing!  Thus a ``raise'' of the
artificial bid is not needed as natural and is instead used to show a
semi-positive hand (a good 6 to 8) with specified shape.  Other
semi-positive hands use an artificial inquiry listed above.  All other
bids show less than semi-positive and are pass/correct or natural
(except for transfers to the majors over 2\nt, and a 2\nt\ response
which shows a weak known two-suiter).

\newpage

\subsection{1\c--1\d--1\h}
1\s\ is a second negative showing 0--4 \hcp; 1\nt\+ is \gf,
transfer-oriented symmetric relay shifted up two steps, not promising
any \ak.

We may eventually modify the two-up relay chart to change some optimizations, but for now it is just shifted up 2.

\subsubsection{1\c--1\d--1\h--1\s}
1\nt\ shows 21--23 \bal.  It's sort of silly that we end up in
1\nt\ with 21 opposite 0 but 2\nt\ with 20 opposite 0, but oh well.
Strong notrump systems are on (responder can certainly have a
game-forcing hand).

2\c\ is an artificial \gf.  Continuations are natural.  (Maybe
responder should (nearly) always bid 2\d?)

2\nt\ shows 24--25 \bal.  3\nt\ is to play, likely a long running suit
and side stoppers.

Suit bids from 2\d\ to 3\c\ are natural.

\subsubsection{1\c--1\d--1\h--1\s--2\x}

% XXX this section made sense before but is way too complicated now--
% how can responder have a game force non-raise

In general, continuations are mostly natural, except that 2\nt\ by
either hand is artificial showing a minimum hand in the context of the
auction.  It is a puppet to 3\c\ when it is responders first bid after 1\s, and
otherwise mostly forcing, but may be passed if a misfit seems likely.

With a bad hand, responder must either
\begin{itemize}
\item pass,
\item jump shift, or
\item bid 2\nt, and over the 3\c\ response, pass or correct to a suit
  below 3\x.
\end{itemize}

An exactly invitational hand with a 6\+-card suit \y\ above \x\ may
bid 2\nt\dots3\y, and an invitational hand with a ``fit'' for \x\ may
bid 2\nt\dots3\x.  The
invitational raise is really just suggesting to play 3\x\ when the
invite is rejected; since we cannot play 2\nt\ this may not be a real
fit and opener should probably bid 3\nt\ even holding hearts.  That
covers the immediate 2\nt\ response by responder.  (The sequence
2\nt\dots3\nt\ is undefined.)

If responder does not make an initial jump shift or 2\nt\ response,
the auction is forcing to 2\nt\ by either player.  Bidding immediately above 2\nt,
or slowly bidding to a new suit above \x at the 3 level establishes a game force.
The artifical 2\nt\ is
not forcing; a minimum hand can pass but will more often pull to a
known fit or to 3\c\ to play in opener's second suit.  A maximum hand
bids above 3\x\ to establish a game force (this is why a hand which is
only invitational must start with 2\nt).

An initial 2\y\ over 2\x\ only shows a 4-card suit.  When we establish
a game force rather than bidding game directly, it tends to show doubt
over strain rather than level.  So 2\x--2\y--3\y\ will usually be a
3-card (\gf) raise.  Opener can make a minimum (3- or) 4-card raise
via 2\x--2\y--2\nt--3\c--3\y.  A jump shift to a new suit after
2\x--2\y\ is a splinter raise.

\subsubsection{1\c--1\d--1\h--1\s--3\c}
If responder does not pass now, a game force is created.

\subsection{Opener's limited artificial rebids over 1\c - 1\d}

\begin{figure}[p]
\begin{tabular}{lllllll}
1\c & 1\d & 1\s &&&& At least one long minor, with a 4-card major possible.\\
&&& 1\nt &&& Semi-positive ask.\\
&&&   &2m && Min, 5\+ m, 4 card M possible.\\
&&&   &   &2M & To play, but opener can raise.\\
&&&   &   &2\nt & Natural, invitational (near max).\\
&&&   &2\h && 4\h, good 5\+ minor, non-min.\\
&&&   &   &2\s & Asks for minor. (2\nt = clubs, 3\c=diamonds).\\
&&&   &   &2\nt & 6\+ \s, forcing.\\
&&&   &   &3m & Pass or correct.\\
&&&   &   &3\h & Natural, invitational.\\
&&&   &2\s && 4\s, good 5+ minor, non-min.\\
&&&   &   &2\nt & Asks for minor.\\
&&&   &   &3m & Pass or correct.\\
&&&   &   &3\h & 6+\h, forcging.\\
&&&   &   &3\s & Natural, invitational.\\
&&&   &2\nt && 5\+ / 4\+ minors, non-min.\\
&&&   &   & 3m & To play.\\
&&&   &   & 3M & 6+M, non forcing but opener can raise.\\
&&&   &   & 3\nt & To play.\\
&&&   &   & 4m & Natural, invitational.\\
&&&   &3m && Good 6\+m, non-min, no 4 card M.\\
&&& 2m &&& Pass or correct.\\
&&& 2\h &&& To play, weaker than semi-positive, but opener can raise.\\
&&& 2\s &&& Artificial semi-positive, 4-4 in majors (possibly 5-4).\\
&&&   & 2\nt && Both minors, min.\\
&&&   & 3m && No major, min.\\
&&&   & 3M && Natural, min, 4 card suit.\\
&&&   & 3\nt && Natural, max, no major.\\
&&&   & higher && Natural, max.\\
&&& 2\nt &&& Weak, artificial, 5\+ / 5\+ in majors?\\
&&& 3\x &&& To play, weak, but opener can raise.\\
&&& 3\nt &&& To play.\\
1\c & 1\d & 1\nt &&&& 16-18 balanced (could have 5 card major).  Strong systems on.
\end{tabular}
\caption{1\c--1\d--(1\s, 1\nt)}
\end{figure}

\begin{figure}[p]
\begin{tabular}{lllllll}
1\c & 1\d & 2\c &&&& Both majors.\\
&&& 2\d &&& Asks for better major.  Semi-positive unless next bid passed.\\
&&&   & 2M && Better major.\\
&&&   &   & 2\s & (over 2\h) Natural, invitational in spades.\\
&&&   &   & 2\nt-3M & Natural, invitational.\\
&&&   &   & 3\h & (over 2\s) Natural, invitational in hearts.\\
&&&   &   & 3\nt & ?\\
&&&   &   & 4M & To play.\\
&&&   &   & others & Splinters.\\
&&& 2M &&& To play.\\
&&& 2\nt &&& Weak, both minors, 5\+, 5\+.\\
&&& 3\c &&& Semi-positive, both minors, 5\+, 5\+.\\
&&& 3\d &&& Natural, weak.\\
&&& 3M &&& Natural, weak with 5 card suit; opener can raise.\\
&&& 3\nt &&& To play.\\
&&& 4m &&& Transfer to corresponding major.\\
&&& 4M &&& To play.\\
1\c & 1\d & 2\d &&&& One major 6\+.\\
&&& 2M &&& Pass or correct.\\
&&& 2\nt &&& Asks strength/suit, semi-positive.\\
&&&   &3\c && Min, \h.\\
&&&   &   &3\d & Puppet to 3\h (after which 3\nt is choice of games).\\
&&&   &   &3\h & To play.\\
&&&   &   &3\s & Natural, 6\+ \s.\\
&&&   &   &3\nt & To play.\\
&&&   &3\d && Min, \s.\\
&&&   &   &3\h & Puppet to 3\s (after which 3\nt is choice of games).\\
&&&   &   &3\s, 3\nt & To play.\\
&&&   &3\h && Max, \h.\\
&&&   &   &3\s & Natural, 6\+ \s.\\
&&&   &   &3\nt & To play.\\
&&&   &   &4m & Natural, invitatinoal.\\
&&&   &3\s && Max, \s.\\
&&&   &   &3\nt & To play.\\
&&&   &   &4m & Natural, invitational.\\
&&&   &   &4M & To play.\\
&&&3\c &&& Weak, long suit \\
&&&3\d &&& Semi-positive, 6\+ \h.\\
&&&3M &&& Weak, long suit.\\
&&&3\nt &&& To play.\\
&&&4m &&& Weak, natural, very long suit.\\
&&&4M &&& Natural, to play, very long suit.\\
\end{tabular}
\caption{1\c-1\d-(2\c,2\d)}
\end{figure}

\begin{figure}[h]
\begin{tabular}{lllllll}
1\c & 1\d & 2M &&&& 5\+ M, 4+ m.\\
&&& 2M+1 &&& Semi-positive ask.\\
&&&   & 2M+2 && Min, \c.\\
&&&   & 2M+3 && Min, \d.\\
&&&   & 3M-1 && Max, \c.\\
&&&   & 3M && Max, \d.\\
&&&   &   & 3\s & (over 2\h) 6\+ spades, forcing if opener max.\\
&&& 2\nt &&& (over 2\h) spades, weak.\\
&&& 3m &&& Pass correct.\\
&&& 3M &&& Natural, invitational.\\
&&& 3oM &&& Weak, long suit.\\
&&& 3\nt &&& To play.\\
1\c & 1\d & 2\nt &&&& 6m322 any quality, good for declaring \nt, almost worth 1\h.\\
&&& 3\c &&& Pass or correct.\\
&&& \hbox to 0 pt{3\d,3\h} &&& Transfer.\\
&&& 3\s &&& Asks for minor (3\nt = \c, 4\c = \d).\\
&&& 3\nt &&& To play.\\
1\c & 1\d & 3\x &&&& Natural, long good suit.\\
1\c & 1\d & 3\nt &&&& To play.\\
\end{tabular}
\caption{1\c-1\d-(2\h+)}
\end{figure}

\newpage

\section{Game forcing responses to 1\c}

\subsection{Basic idea}
A \gf\ response requires 2 \ak; see \ref{controls} Controls.  The 1\c\ opener
is now the \emph{relayer} or \emph{asker} and will continue making the
cheapest available bid (with two exceptions, see below) until able to
place the contract.  The other hand is called the \emph{relay
  responder} and makes a sequence of descriptive bids (separated by
semicolons here).  Ideally, the relayer will be the declarer and the
defenders will have no information about the closed hand; hence
``transfer-oriented.''

\subsection{Signoffs}
If the relayer bids 3\nt, that is always to play.  A bid of 4\d\ by
the relayer is the \emph{end signal}: responder bids 4\h\ whereupon
the relayer places the contract.  But with a very good hand responder
can bid over 3\nt\ or 4\d\ showing controls, starting with 4\c/4\s\ =
4 \ak.  (See \ref{controls} Controls.)  When the next step is 3\nt\ or 4\d,
relayer can continue relays by bidding one step higher instead.

\subsubsection{Slam Invites}

\subsubsection{Quantitative Invites}

\subsection{Responder's bids}
The responding hand is described in four stages: shape, range,
controls, and denial cuebidding (location of specific honors).
Sometimes a single bid belongs to two or more stages, thanks to a
technique called \emph{zooming}: when responder would make the highest
possible bid at the end of one stage, responder also begins the next
stage as though the relayer had made a relay ask one step below that
bid.  For example, the sequence 1\s; 3\d\ just shows 3424 shape, but
1\s; 3\h\ shows 4243 shape and a minimum, 1\s; 3\s\ shows 4243 shape
and a maximum with 3 \ak, etc., because there is no shape assigned to
1\s; 3\s.

\subsection{Shape}
Hands are divided into one-suited, two-suited, three-suited and
balanced (for our purposes defined as 4432 or 4333) shapes.  A
``suit'' is a suit of at least 4 cards, provided the hand is not
balanced.  Every shape that does not have an 8-card suit or 12 cards
in two suits is shown exactly, except that the 2- and 3-card suits of
a 7321-type hand are not distinguished.  With extreme shape, make the
smallest lie possible.

\subsubsection{One-suiters}
First bid your suit using the scheme 1\h\ = \s, 1\nt\ = \h, 2\c\ = \d,
2\d\ = \c.  (This ensures that the relayer will play the hand if we
play in responder's long suit, provided it is not \h.)  Your next bid
will be 2\s\ or higher.  Look at your length in your other three suits
and see if you have a unique shortest suit.  If so, bid 2\s\ if it is
the highest side suit, bid 2\nt\ if it is the middle side suit, and do
nothing for now (a kind of zoom) if it is the lowest side suit; then
show your shape: 3\d\ = 5332, 3\h\ = 6331, 3\s\ = 7330, \zoom{3\nt} =
7321.  The underline is a reminder to zoom to range and controls,
since 7321 is the highest possible shape on this sequence.  If you do
not have a unique shortest suit, your shape must be 6322 or 7222, and
the sequences are 2\s; 3\c; 3\h = 6223, 2\s; 3\c; \zoom{3\s} = 6232,
3\c; 3\h = 7222, 3\c; \zoom{3\s} = 6322.  (How do you remember these
sequences?  The general rule is increasing numeric order; the general
exception is that when the highest bid would then show a rarer shape
than the bid one lower, we switch them, because zooming saves
\emph{two} steps.)

\subsubsection{Two-suiters}
Generally begin by bidding both your suits according to the one-suited
scheme.  However, partner's relay asks will prevent you from making
two consecutive bids!  So, there are special sequences for showing
\d+\h\ or \c+\d: with \d+\h\ bid 1\s; 2\c, and with \c+\d\ do nothing
for now; your first bid will be 2\h\ or higher, showing a two-suiter
with the minors.  There's one other slight adjustment: your next bid
might be 2\h, but you can't do that if you bid 2\d\ showing \c.  So
showing a major and then bidding 2\h\ shows \c, also, as though it
were 1\h/1\nt; 2\d; 2\h.  (We didn't use the 2\h\ rebid for
one-suiters, so this is OK.)

Suppose first that one of your suits is only 4 cards long.  Bid
2\h\ now if it's the higher one.  Next bidding 3\c\ shows 5422 and
\zoom{4\c} shows 7411.  Otherwise, one of your short suits is shorter
than the other; bid 2\nt\ if it's the higher one.  Then show your
shape: 3\d\ = 5431, 3\h\ = 6421, 3\s\ = 6430, 3\nt\ = 7420 (zoom from
3\nt\ only if you bid 2\nt\ first).  (Generally when $x$ shows two
equal lengths, $x-1$ shows ``high shortage.'')

If instead both your suits have at least 5 cards, bid 2\s.  Now
3\d\ shows singletons in both short suits, followed by 3\s\ = 5611,
\zoom{3\nt} = 6511.  Otherwise, bid 3\c\ if your higher short suit is
shorter, then 3\h\ = 5521, 3\s\ = 5530, 3\nt\ = 5620, \zoom{4\c} =
6520.  (The first two digits in these numbers are the lengths of your
long suits, with the higher one first; the second two are the lengths
of your short suits with the longer one first.)

\subsubsection{4432 and 4333 shapes}
With these balanced shapes, start with 1\s\ (so that relayer will
declare a notrump contract).  Your next bid will be 2\d\ or higher.
With 4432, consider your two 4-card suits; bid 2\d\ if they are the
same color, 2\h\ if they are the same rank, and do nothing for now if
they are the same shape; then bid your doubleton: 2\nt\ = \s, 3\c\ =
\c, 3\d\ = \d, \zoom{3\h} = \h.  With 4333, bid 2\d\ with a 4-card
major, then 2\s\ always, then 3\c\ with long \c\ or \h\ and \zoom{3\d}
with long \d\ or \s.

\subsubsection{Three-suiters}
These are last because they live in the gaps formed by the rest of the
system.  The two unused sequences are 1\h; 1\nt; 2\d\ and 2\c; 2\h.
The first shows a three-suiter with both majors, the second a
three-suiter with both minors.  After one of these starts, there are
only 8 possible shapes, so the follow-ups are simple and step-based:
bid the cheapest step first if you have a high short suit (\s\ or \d),
then bid in steps showing the lengths of your long suits in
\s\h\d\c\ order (444, 445, 454, 544---and zoom with 544).

\subsection{Range}
Bid the cheapest step with a minimum, otherwise zoom on to controls.
A minimum is a hand with fewer than 12 \hcp\ or fewer than 3 \ak.  A
passed hand is always a minimum and skips this stage.  

\subsection{Controls} \label{controls}
We define \ak\ as a measure of controls: an Ace is worth 2 \ak, a King
1 \ak.  We don't count any \ak\ for shortness, since relayer already
knows responder's shape.  \ak\ points are shown in steps, starting
with 2 (and zooming with 5) if responder has shown a minimum, 3 (never
zooming) if a maximum.

\subsection{Denial cuebidding}
Sort your suits by length with the longer suits first, breaking ties
by putting higher ranking suits before lower ones.  You will cycle
through the suits in this order, looking at each suit a number of
times depending on its length: three times for suits of at least 4
cards, twice for 3-card suits, once for 2-card suits, and never for
singletons or voids.  The first time you look at a suit, check whether
it contains exactly one of the Ace or King; the second time, check
whether it contains the Queen; the third time, check whether it
contains the Jack.  (There is an exception for suits of 6 or more
cards described below.)  So, once you have all these ``yes'' or ``no''
answers, what do you \emph{bid}?  If the answer to the first question
is ``no,'' bid the cheapest step.  In general, skip a number of steps
equal to the number of ``yes'' answers.  If relayer continues, start
again where you left off, with the answer after the first ``no.''  In
these auctions, any non-relay bid by the relayer is to play, as is any
bid at the slam level; the highest relay ask is 5\nt.

\subsubsection{Long suit exception}
When checking a suit of 6 or more cards, replace the first two checks
with the following: first check whether it contains at least two of
the Ace, King and Queen, then check whether it contains an odd number
(the larger possibility given the answer to the first check).

\subsubsection{Josh's Law}

"Josh�s Law: Whenever information regarding high cards or their location is absolutely known from previous bidding, then that information is skipped even if it hasn�t been asked for yet. This can take many forms, such as knowing responder can have no more controls, or high card points, or can even pertain to distribution." \cite{TOSR}

We do not currently play Josh's Law, but may eventually.

\subsection{Examples}  In each example North deals and opens 1\c.
\[
\begin{tabular}{ll|ll|lll|ll|l}
1.
&\s\ KQ73&1\c&2\c&\d&                     \qquad 2.&\s\ A9&1\c&1\s&balanced or \d+\h\\
&\h\ AKJ2&2\d&2\h&three-suited with \c+\d&         &\h\ AQ1072&1\nt&2\h&same rank\\
&\d\ Q10&2\s&3\c&4144&                             &\d\ AQJ&2\s&2\nt&2344\\
&\c\ Q107&3\d&3\nt&maximum, 4 \ak&                 &\c\ AK9&3\c&3\d&minimum\\
&&4\d&4\h&end signal&                              &&3\h&3\s&2 \ak\\
&\s\ J654&4\s&---&9 \ak\ is not enough&            &\s\ J2&4\c&4\h&\d K, no \c K\\
&\h\ 4&&&  for slam without&                     &\h\ K98&4\s&5\c&\h K, no \s K\\
&\d\ K954&&&  more distribution.&                &\d\ K642&5\d&5\h&no \d Q\\
&\c\ AKJ6&&&&                                      &\c\ J732&5\s&5\nt&no \c Q\\
\multicolumn{5}{}&&                                &&6\nt&---&7\nt\ is at best\\
\multicolumn{5}{}&&                                &&&&  on a finesse.
\end{tabular}
\]

\subsection{Relay Breaks}

If opener makes a bid which is not the relay bid and is not a signoff bid, it is a relay break.  The meaning is determined by the following list.

\begin{enumerate}
\item 10-shape reverse relay.  If this is opener's first rebid above a 1\h, 1\s or 1\nt bid by responder, a relay break begins a 10-shape reverse relay sequence. 
\item 4-shape reverse relay.  If this is opener's first rebid above a 2\c or 2\d bid by responder, a relay break begins a 4-shape reverse relay sequence.
\item Stopper ask.  If responder has $n$ known 2 or 3 card suits, the first $n$ relay breaks are stopper asks, if below 3\nt.
\item Exclusion.  If responder has finished showing shape with 3\h or lower, then 4\c is the exclusion relay.
\item RKC ask.  If none of the above relay breaks applies and the bid is not a signoff or a slam invite, then it is a Roman keycard ask.
\end{enumerate}

\subsubsection{10-shape Reverse Relay}

\subsubsection{4-shape Reverse Relay}

\subsubsection{Stopper Ask}

\subsubsection{Exclusion}

\subsubsection{RKC Ask}

\section{\tf\ seat 1\c\ openings} \label{tf1c}

As normal, an opening of 1\c\ in \tf\ seat promises 2\hcp\ more than in \fs.  Thus all of responder's point ranges are shifted down by 2 points.

Responder now needs 1-4 \ak\ and 7-10 \hcp\ for a positive response.

There is no distinction between minimum and maximum for a responder's hand: after showing shape, the next question is number of \ak.

\section{Interference Over 1\c}

\subsection{Low-level auctions}

If they bid 1\h\ over our 1\c\ we ignore it---\p\ replaces 1\d\ and
\x\ replaces 1\h\ (by either player).

If they double or bid 1\d, responder may:
\begin{itemize}
\item Make the same relay response they would otherwise, game forcing
  as usual.  We ignore the \x/1\d.
\item Bid 1\d\ or double it, showing the weakest range (0--6).  Even
  though this is not the normal range for a 1\d\ bid, we pretend the
  auction has gone 1\c--1\d\ as far as our systems are concerned.
  1\h\ now shows 21\+, over which 1\s\ is 0--3.  Other rebids show
  16--20.
\item Pass (with \c\ length) or redouble (for takeout) over a double,
  or pass 1\d.  This shows a semi-positive.  We bid as though they
  opened 1\c/1\d: double/redouble by opener is takeout, the rest
  natural.
\end{itemize}

\begin{figure}[p]
\begin{tabular}{cccl}
1\c & (x) & -- & 7-8\hcp, some \c\ length\\
1\c & (x) & xx & 7-8\hcp, takeout of \c\\
1\c & (x) & 1\d & 0-6\hcp\\
1\c & (x) & 1\h + & normal relays\\
1\c & (1\d) & -- & semi-positive, non-\gf\\
1\c & (1\d) & x & 0-6\hcp \\
1\c & (1\d) & 1\h + & normal relays\\
1\c & (1\h) & -- & Non-\gf\ (opener's \x\ = 1\h, strong; others as over 1\c--1\h)\\
1\c & (1\h) & x & Would have bid 1\h\ (normal relays)\\
1\c & (1\h) & 1\s + & normal relays\\
1\c & (1\s,2\c,2\d) & -- & 0-5\hcp\ or penalty pass or 6-8\hcp but not able to double.\\
1\c & (1\s,2\c,2\d) & x & 6-8 \hcp, 3-4 cards in opponents suit; flattish with penalty interest or \gf without stopper.\\
1\c & (1\s,2\c,2\d) & cue & \gf, takeout, decent support in unbid majors.\\
1\c & (1\s,2\c,2\d) & suit jump & 6-8\hcp\ with 6\+ (or good 5) in suit.\\
1\c & (1\s,2\c,2\d) & bid & Natural, \gf.\\
1\c & (2M) & -- & Nothing to say or penalty pass. \\
1\c & (2M) & x & 6-8\hcp, possibly more. \\
1\c & (2M) & cue & Looking for stopper for 3\nt. \\
1\c & (2\h) & 3\s & ? \\
1\c & (2M) & 4m & Leaping Michaels. \\
1\c & (2M) & 4M & Both minors. \\
1\c & (2M) & bid & Natural, \gf.\\
1\c & (3\x) & -- & Nothing to say or penalty pass.\\
1\c & (3\x) & x & Take-out oriented, \gf.\\
1\c & (3\x) & 4\x & Huge hand, takeout.\\
1\c & (3\x) & suit jump & \gf, playable opposite no support.\\
1\c & (3\x) & bid & Natural, \gf.\\
1\c & (4\c+) & -- & Nothing to say or penalty pass.\\
1\c & (4\c+) & x & Game values, cooperative.\\
1\c & (4\c+) & cue & Huge hand, takeout.\\
1\c & (4\c+) & \nt & Huge two suiter.\\
1\c & (4\c+) & bid & Natural.\\
\end{tabular}

\begin{tabular}{cccccl}
1\c & (--) & 1\d & (1M) & -- & forcing a double, could be penalty. \\
1\c & (--) & 1\d & (1M) & x & Takeout. (later jump cue asks for stopper)\\
1\c & (--) & 1\d & (1M) & 1\nt & 16-18\hcp, balanced.  Stopper and a source of tricks. \\
1\c & (--) & 1\d & (1M) & 2\nt & 21-22\hcp, balanced.  Stopper and a source of tricks. \\
1\c & (--) & 1\d & (1M) & jump shift & Natural, \gf \\
1\c & (--) & 1\d & (1M) & cue & Big 2/3 suit takeout. \\
1\c & (--) & 1\d & (1M) & jump cue & Natural (psych exposer). \\
1\c & (--) & 1\d & (x = \d) & xx & penalty (to play).\\
1\c & (--) & 1\d & (x = \d) & -- & strong, takeout-ish, relays on if passed to responder.\\
1\c & (--) & 1\d & (x = \art) & bid & Natural.\\
1\c & (--) & 1\d & (x = \art) & jump & Strong.\\
1\c & (--) & 1\d & (x = suits) & cue & Takeout.\\
1\c & (--) & 1\d & (x = suits) & jump cue & Psych-exposer.\\
\end{tabular}
\end{figure}

\begin{figure}[h]

\centering
\begin{minipage}{\textwidth}
\begin{tabular}{c}

\begin{tabular}{l|l|l|l|l|}
& 1-suiter & 2-suiter & 3-suiter & balanced \\
1\h & \s & \s & \s\hsmash{\footnote{With \s\d\c\ bid 2\c\ first}} & -- \\
1\s & -- & \h\d\ (2\c\ next) & -- & balanced \\
1\nt & \h & \h & \h\hsmash{\footnote{With \h\d\c\ bid 2\c\ first}} & -- \\
2\c & \d & \d\hsmash{\footnote{With \d\c\ bid 2\h\+ directly}} & \d & -- \\
2\d & \c & \c\ (non-reverse) & \h+\s+(\d\ or \c) & \textbf{BAL} \\
2\h & -- & reverse (45\+); \c & \c\+\d+(\s\ or \h) & \smash{\vdots} \\
2\s & high shortage & 5\+5\+ & \textbf{3-SUITER} & \\
2\nt & mid shortage & high shortage & \textbf{3-SUITER} & \\
3\c & equal shortage & 5422 / high shortage && \\
3\d & 5332 & 5431 / equal shortage &&\\
3\h & 6331 / 6223 / 7222\hsmash{\footnote{6223 is shown via 2\s\ and 3\c\ while 7222 is shown via 3\c\ only.}} & 6421 / 5521&& \\
3\s & 7330 / \zoom{6232} / \zoom{6322}\hsmash{\footnote{6232 is shown via 2\s\ and 3\c\ while 6322 is shown via 3\c\ only.}} & 6430 / 5530 / 5611&& \\
3\nt & \zoom{7(32)1} & 7420 / 5620 / \zoom{6511}&& \\
4\c & \smash{\vdots} & \zoom{7411} / \zoom{6520}&& \\
4\d & & \smash{\vdots}&&\\
\end{tabular} \\
\\
\hline
\begin{tabular}{c|c}
&\\
\textbf{3-SUITER} & \textbf{BALANCED}\\

\begin{tabular}{lll}
1&&high shortage (then start at 2)\\
2&&5440 (then ascending numeric)\\
&a&4450 \\
&b&4540 \\
&c&\zoom{5440}\\
3&&\zoom{4441}\\
4&&\smash{\vdots}\\
\\
\\
\end{tabular} &

\begin{tabular}{l|l|l|}
& 4432 & 4333 \\
2\d & color & 4-card major \\
2\h & rank & -- \\
2\s & -- & always \\
2\nt & 2\s & -- \\
3\c & 2\c & 3433 \\
3\d & 2\d & \zoom{4333} \\
3\h & \zoom{2\h} & \smash{\vdots} \\
3\s & \smash{\vdots} &
\end{tabular}\\
\\
\end{tabular} \\
\hline
\begin{tabular}{c|c}
\\
\textbf{10-SHAPE RR} / 1\h,1\s,1\nt & \textbf{4-SHAPE RR} / 2\c, 2\d \\
\begin{tabular}{lll}
1 && 4441 / 4450 / 4540 / 5440 (\textbf{4-SHAPE RR})\\
2 && 4531 / 3451 / 5341\\
&a& 4531\\
&b& 3451\\
&c& \zoom{5341}\\
3 && 4351\\
4 && 3541\\
5 && \zoom{5431}\\
6 && \smash{\vdots} \quad\quad \textbf{Hearts primary with known \h\ fit}
\end{tabular} &
\begin{tabular}{lll}
1&& 4450 / 4540 / 5440\\
&a& 4450\\
&b& 4540\\
&c& \zoom{5440}\\
2&& \zoom{4441}\\
3&& \smash{\vdots}\\
\\
\\
\multicolumn{3}{l}{\textbf{No zoom past} 3\nt\ \textbf{without known fit}}\\
\end{tabular}
\end{tabular}
\end{tabular}
\end{minipage}
\caption{Relay Summary}
\end{figure}

\chapter{The 1\d\ opening}

The 1\d\ opening, in \fs, shows one of
\begin{itemize}
\item 11--13 \bal,
\item good 10--bad 15 three-suited (4441, 5431, 5440), without a 5-card major,
\item 10--14 both minors (55 or longer).
\end{itemize}
The point ranges are 1 point higher in \tf.

Initial responses are natural.
\begin{itemize}
\item[\p] 0--5 \hcp\ or so, usually balanced or both minors or a long
  minor not good enough for a weak jump shift to the three level.  We
  would respond 1\M\ on any hand which has somewhere to go if opener
  rebids 1\nt.
\item[1\h/1\s] 4\+-card major, 0\+ \hcp, may have longer minor even if \gf\ strength.
\item[1\nt] about 6--11 \hcp, no 4-card major, no game interest opposite 11--13 \bal.
\item[2\c/2\d] 4\+-card minor, maybe \bal, 12\+ \hcp, no cheaper 4-card suit.
\item[2\h/2\s] weak jump shift, 6\+-card suit, 0--9 \hcp\ or so, no
  game interest unless opener has a big fit.  This bid is mandatory
  since rebidding 2\M\ after 1\M\ would be invitational.
\item[2\nt] balanced \gf, 13--16 \hcp, asking for shortness.
\item[3\c/3\d] weak jump shift, 7\+-card suit, not invitational opposite 11--13 \bal.
\item[3\nt] to play (not interested in 4\M/5m).
\end{itemize}
Everything else is preemptive, as though responder was opening.

In Midchart events, we do not play weak jump shifts to 3\c/3\d\ but instead:
\begin{itemize}
\item[2\nt] balanced \gf, 13--16 \hcp, \OR\ weak jump shift in a minor
\item[3\c/3\d] 7-card suit to two of the top three honors or AJT or KJT; no side Ace or King
\end{itemize}

XXX Passed hand bidding?

\section{1\d--1\h/1\s}

Opener has many raises available.  We raise on 3-card support when the
hand contains a singleton (or void) or small doubleton, except that
opener rebids 1\s\ in preference to raising 1\h\ with 3.

\subsection{Opener does not raise}

Over 1\h, opener's first priority is to show a 4-card spade suit by
bidding 1\s.  We basically play our ordinary two-way new minor forcing
here.

\begin{figure}[h]
\begin{center}
\begin{tabular}{llllll}
1\d&1\h&1\s&1\nt&&\nf, 6--11 \hcp, 4--5 \h s, no game interest opposite 11--13 bal\\
&&&&2\c/2\d&\nf, \nat, 5-card suit\\
&&&&2\h&3-card support, does not promise extras XXX always minimum?\\
&&&&2\s&maximum, 4144/40(54)\\
&&&&2\nt&maximum, 43(15).  3\c\ = pass/correct, 3\d\ asks, 3\h\ to play.\\
&&&2\c&&puppet to 2\d\ to play or for invite (then 3\nt\ = choice of 3\nt/4\h)\\
&&&2\d&&artificial game force; opener rebids naturally\\
&&&2\h&&invite with 6 \h s\\
&&&2\s&&4 \s s, not invitational\\
&&&4\c&&splinter (\s\ agreed)\\
&&&4\d&&splinter (\s\ agreed)
\end{tabular}
\end{center}
\caption{1\d--1\h--1\s}
\end{figure}

Otherwise, opener rebids 1\nt\ with 11--13 balanced or a
three-suiter (5431 or 4441) of equivalent playing strength with a
singleton in responder's major.  See below for transfers.

The remaining hands are either 55 in the minors or too strong to
respond 1\nt.  They are handled as shown below.

\begin{figure}[h]
\begin{center}
\begin{tabular}{lllllll}
1\d&1\M&2\c&&&denies 3-card support; minimum 5\+5\+ minors \OR\\
   &   &   &&&maximum three-suiter short in responder's major except 1453 (after 1\s)\\
   &   &   &2\d&&\nf\ preference, not invitational\\
   &   &   &2\M&&invitational with a 6-card suit\\
   &   &   &2\OM&&fourth suit forcing to game, relays\\
   &   &   &2\nt&&invitation, no fit\\
   &   &   &    &3\c/3\d&65 minimum\\
   &   &   &    &3\h&maximum with 4 \h s\\
   &   &   &    &3\nt&maximum\\
   &   &   &3\c/3\d&&invitation, 3-card support\\
   &   &   &       &3\h&maximum with 4 \h s\\
   &   &   &       &3\M&stopper ask in \OM\\
   &   &   &       &3\nt&maximum\\
   &   &   &3\M&&natural, \gf\\
   &   &   &4\c/4\d&&natural slam try\\
   &   &3\c&&&maximum, 5\+5\+ minors XXX artificial follow-ups probably better\\
   &   &   &3\d&&to play\\
   &   &   &3\M&&asking for a raise on Hx support\\
   &   &   &3\OM&&asking for a stopper or shortness in \OM\\
   &   &   &    &3\nt&stopper\\
   &   &   &    &4\c&shortness \\
   &   &   &    &5\c&no stopper or shortness, pass/correct\\
   &1\s&2\d&&&maximum, 1453 exactly.  2\h\ through 3\c\ are signoffs, higher bids are \gf.
\end{tabular}
\end{center}
\caption{Opener's rebid with 5\+5\+ minors or a maximum misfit}
\end{figure}

\subsection{Transfers after 1\d--1M--1\nt} 

Responder's rebid of 2\M\ always shows an invite with 6 cards in \M.
2\c\ handles all other invites, plus some other hands; other bids
through 3\c\ are transfers, except 2\s\ is always natural; and higher
bids are \gf\ and mostly natural (except 3\OM\ is a splinter).

After 1\d--1\M--1\nt:

\begin{enumerate}
\item[2\c] puppet to 2\d, for \d\ signoff, most invites, or simple 3\nt/4\M\ choice-of-games
\item[2\M-1] transfer to \M, \gf\ with 5 \M s (not 5 \s s 4\+ \h s)
\item[2\M] invite with 6 \M s
\item[2\d] if \M=\s, shows 5\+ \s s, 4\+ \h s, weak or \gf.  Opener should
  give preference, assuming a weak hand.
\item[2\s] if \M=\h, natural, showing exactly 44 in the majors, \gf
\item[2\nt] transfer to 3\c, for \c\ signoff or \gf\ with 5\+ \c s, usually exactly 4 \M s.  Next 3\d\ LMM with 6\+ \c s, or 3\h\+ SMM.
\item[3\c] transfer to 3\d, for (longer) \d\ signoff or \gf\ with 5 \d s, usually exactly 4 \M s.  Next 3\h\+ SMM.
\item[3\d] \gf\ with 6\+ \d s, LMM
\item[3\M] \nat\ \gf, single-suited, but not a solid suit.  3\nt, 4\M\ are to play, others cues showing Hx+ support.
\item[3\OM] (41)44 \gf
\item[3\nt] to play
\item[4m] natural, 6 \M s, 5 ms
\item[4\M] to play
\item[4\nt] quantitative invite (quite likely (43)33?)
\end{enumerate}

\subsubsection{2\c}

After 1\d--1\M--1\nt--2\c--2\d\ (forced), jumps are choice-of-games and
other bids natural invites:

\begin{enumerate}
\item[\p] \d\ signoff
\item[2\M] invite with exactly 5 \M s
\item[2\h] if \M=\s, invite with 5 \s s 4\+ \h s
\item[2\s] if \M=\h, invite with exactly 44 in the majors\footnote{
  Even in our style, we might prefer to play 2\s\ in the 4-3 fit to 2\nt.
}
\item[2\nt] natural invite (with exactly 4 \M s)
\item[3\c] natural invite (5\+ \c)
\item[3\d] natural invite (5\+ \d)
\item[3\h] if \M=\s, strong invite with 55 in the majors; if \M=\h, strong invite with 5 \s s 6 \h s
\item[3\s] XXX ???
\item[3\nt] choice of 3\nt\ or 4\y, with 5 \y s but leaving the decision up to opener
\end{enumerate}

XXX There are a bunch of 4-level bids available, including 4\y

XXX Maybe the 3\s\ could also be a slammish checkback?

In the sequence 1\d--1\M--1\nt--2\c--2\d--2\M, it is almost mandatory
for opener to raise to at least 3\M\ with 3-card support, as responder
may have an invitational 55 which improves greatly with a known fit.
(With a maximum and a fit, opener of course bids 4\M.)  If opener bids
anything else below game, rejecting the invite without a fit,
responder might then run to a second 5-card suit at the 3-level.

\subsubsection{Transfers}

The transfer 1\d--1\M--1\nt--2\M-1 is used with most \gf\ hands with a
5-card major.  The exceptions are

\begin{itemize}
\item hands with a solid suit, which jump to 3\M;
\item hands which are willing to play 3\nt\ opposite some openers with
  3-card support;
\item hands with at least 5 spades and at least 4 hearts, which
  transfer to \h\ instead;
\item hands with at least 65 (either way) in a major and a minor,
  which jump to 4m (with a 6-card major) or transfer to the minor
  (with a 5-card major).
\end{itemize}

After 1\d--1\M--1\nt--2\M-1--2\M:

\begin{itemize}
\item[2\s] (\M=\h) natural \gf.  Opener usually rebids 2\nt\ to hear
  more about responder's hand (3m = fragment, typically 45(31);
  3\h\ = 6 \h s; 3\s = 5 \s s 6 \h s)
\item[2\nt] \gf\ transfer, 4--5 \c.  With 5 \c, 3\d\ LMM next, otherwise 3\h\+ SMM.
\item[3\c] \gf\ transfer, 4 \d, next 3\h\+ SMM.
\item[3\d] \gf, 5 \d, LMM.
\item[3\h] (\M=\s) XXX natural \gf\ probably, but what shape?  7411?
\item[3\M] \gf, sets trumps, solid suit.  Frivolous 3\nt\ is on.
\item[3\nt] \nf, choice of games, but demanding that opener raise with
  3-card support.  Opener could cue with a great hand, I guess.
\item[Jumps] self-splinters.
\item[4\M] \nf\ mild slam try.
\item[4\M+1] RKC for \M.
\end{itemize}

After we show both majors, bidding is generally natural.  For example
after 1\d--1\s--1\nt--2\d--2\h/2\s:

\begin{itemize}
\item[2\s] \gf\ 64 majors
\item[2\nt] XXX ???
\item[3\c] fragment, usually 5413, \gf
\item[3\d] fragment, usually 5431, \gf
\item[3\h] \gf\ 55 majors
\item[3\s] XXX ???
\item[3\nt] 5422 \nf
\end{itemize}

\subsubsection{1\h\dots2\s}

This slightly silly sequence promises exactly 44 in the majors.  If
responder were balanced he would just bid some number of \nt\ instead,
so it also implies shortness in one of the minors.  Opener can bid
3\nt\ with wasted stoppers in both minors, or 2\nt\ to hear more.
After 1\d--1\h--1\nt--2\s--2\nt:

\begin{itemize}
\item[3\c] 4414
\item[3\d] 4441
\item[3\h] 4405
\item[3\s] 4450
\end{itemize}

\chapter{The 1\h\ and 1\s\ openings}

We open 1\h\ or 1\s\ with a 5-card major and 10--14 \hcp\ in \fs,
12--16 \hcp\ in \tf\ (though we could occasionally have fewer than
12).  We open 1\h\ or 1\s\ even with a 6-card minor if the hand is
minimum.  We normally do not open 1\nt\ with a 5-card major.

Our non-jump responses are fairly similar to standard 2/1.  1\s\ over
1\h\ is natural and forcing with 4\+\ spades, and 1\nt\ is a forcing
notrump.  However, neither of these bids promises any values.  With a
non-invitational 3-card raise, we make a delayed raise through 1\nt.
In repsonse to 1\h\ with a hand of invitational strength, exactly 4
spades, and no heart fit, we start with the forcing notrump and then
bid 2\s\ on the next round; we do not use the common treatment of this
sequence as a strong raise of opener's second suit.  We also bid a
forcing 1\nt\ over 1\s\ with 1444 shape.  A jump into a new suit on
the second round (except 1\s--1\nt--2\d--4\c, which passes 3\nt\ with
no known major suit fit) reveals this hand type; the auction
1\s--1\nt--2\s--3\nt\ is also possible.

A 2/1 bid is forcing to game and is almost always a 5-card suit: the
only common exception is (4432) shape with 2 cards in opener's major.
A rebid by opener above 2 of the opened major shows extra values.

Our raise structure is less standard.  We play a raise to 2 as an
\emph{invitational} 3-card raise, about 11--13 support points.  A jump
to 3 is a mixed raise, about 8--10 points with 4-card support; opener
should usually not bid game without extra shape.  A jump to 4 is a
two-way bid, showing either a hand with a fit and game values that
sees little chance of slam, or a hand with enough shape to want to
play in game even without values (5-card support and a singleton or
4-card support and a void).  We have two ranges of splinters, shown in
Figure \ref{fig:1M-splinters}.  This leaves the invitational 4-card
raises and slammish 3-card raises.  We handle both through the
2\nt\ gadget, described in Figure \ref{fig:1M-2NT}.

The jump shift 1\h--2\s\ is weak; jump shifts to the three level are
natural and invitational.

\begin{figure}
\begin{center}
\begin{tabular}{lllll}
1\h
&3\s &&Lower-range splinter in an undisclosed suit\\
&    &3\nt&Asks which suit.  4\c\ = \c, 4\d\ = \d, 4\h\ = \s.\\
&3\nt&&Upper-range splinter in \s\\
&4\c &&Upper-range splinter in \c\\
&4\d &&Upper-range splinter in \d\\
1\s
&3\nt&&Lower-range splinter in an undisclosed suit\\
&    &4\c&Asks which suit.  4\d\ = \d, 4\h\ = \h, 4\s\ = \c.\\
&4\c &&Upper-range splinter in \c\ or \h\\
&    &4\d&Asks which suit.  4\h\ = \h, 4\s\ = \c.\\
&4\d &&Upper-range splinter in \d\\
&\hbox to 0pt{\hss(}4\h&&To play)\\
\end{tabular}
\end{center}
\caption{\label{fig:1M-splinters}Splinter raises of a 1\h\ or 1\s\ opening.}
\end{figure}

\begin{figure}
\begin{center}
\begin{tabular}{llllllll}
1\M&2\nt\\
&&3\c &&&&&minimum without shortness or with a singleton \OM,\\
&&    &&&&&\quad or a maximum without shortness not interested in 3\nt,\\
&&    &&&&&\quad or any hand with a void\\
&&    &3\d &&&&Responder wants to go to game opposite any hand\\
&&    &    &&&&\quad except possibly the minimum without shortness.\\
&&    &    &3\M &&&minimum without shortness\\
&&    &    &3\OM&&&\OM\ shortness, any strength\\
&&    &    &    &3\nt&&Asking for further description.\\
&&    &    &    &    &4\c&maximum, \OM\ void\\
&&    &    &    &    &4\d&minimum, \OM\ void\\
&&    &    &    &    &4\h&minimum but game-going, \OM\ singleton.\\
&&    &    &    &    &4\s&dead minimum, \OM\ singleton.\\
&&    &    &    &    &   &\quad Combined with the previous step when \M\ = \h.\\
&&    &    &3\nt&&&minimum, \c/\d\ void\\
&&    &    &    &4\c &&Asks which minor.  4\d\ = \d, 4\h\ = \c.\\
&&    &    &4\c &&&maximum, \c\ void\\
&&    &    &4\d &&&maximum, \d\ void\\
&&    &    &4\M &&&maximum without shortness not interested in 3\nt\\
&&    &3\M &&&&Responder wants to go to game only opposite a maximum. (\nf)\\
&&    &    &&&&\quad 3\nt, 4\c, 4\d\ continuations as above; 4\M\ to play.\\
&&3\d &&&&&minimum, \d\ singleton\\
&&3\h &&&&&minimum, \c\ singleton\\
&&3\s &&&&&maximum, any singleton\\
&&    &3\nt&&&&Asking for the singleton.  4\c\ = \c, 4\d\ = \d, 4\h\ = \OM.\\\
&&3\nt&&&&&maximum without shortness interested in 3\nt\\
&&4\c &&&&&maximum, 5-card \c\ suit\\
&&4\d &&&&&maximum, 5-card \d\ suit\\
&&4\h &&&&&maximum, 5-card \OM\ suit or (74)11\\
\end{tabular}
\end{center}
\caption{\label{fig:1M-2NT}1M--2\nt}
\end{figure}

\chapter{The 1\nt\ opening \cite{Heeman}}

The 1\nt\ opening shows 14-16 in \fs and 15-17 in \tf.  In addition, the systems used here are also used over 1\c-1\d-1\nt, over 1\c-1\d-1\h-1\s-1\nt, and direct seat 1\nt\ overcalls.

We open 1\nt with 4333, 4432, 5m332, 5m422 and some 6m322 and 4441 shapes.

XXX When are 5 card majors allowed?  Heeman doesn't handle them all that well.

\begin{figure}[ht]
\begin{center}
\begin{tabular}{lll}

2\c & Puppet to 2\d & a) signoff in diamonds \\
&& b) \inv+ with a 5-card major \\
&& c) slammish with a good long major \\
&& d) weak or invitational with 5-4 or 5-5 in the minors \\
&& e) slammish with a long minor \\
2\d & Transfer to hearts & a) signoff in hearts \\
&& b) exactly four hearts, invitational or better \\
&& c) slammish with long, weak hearts \\
&& d) mildly slammish 4=4=4=1 \\
&&e) invitational or better with at least 4-4 in the majors \\
2\h & Transfer to spades & a) signoff in spades \\
&& b) exactly four spades, invitational or better \\
&& c) slammish with long, weak spades \\
2\s & Asks strength & a) weak with clubs \\
&& b) balanced and invitational with no four-card major \\
&& c) slammish with 5-4 or longer in the minors \\
&& d) balanced, slammish (CONFIT) \\
2\nt & Asks weak doubleton & a) invitational with a long minor\\
&& b) slammish with a long minor \\
&& c) a game hand with a long solid major (rare) \\
3\c & Puppet to 3\d & a) weak with diamonds \\
&& b) slammish 4441 (very slammish if 4=4=4=1) hand \\
3\d & Multi-invitational & long major with game-invitational values \\
3\h & Spliter & 31(54), game forcing \\
3\s & Spliter & 13(54), game forcing \\
3\nt & To play \\
4\c & Gerber \\
4\d,4\h & Texas transfer \\
4\s & Choice of minors & At least 5-5 in the minors \\
4\nt & Natural slam invite & usually 4333 \\
5\c,5\d & To play \\
\end{tabular}
\end{center}
\caption{1\nt\ responses}
\end{figure}

For further details see \cite{Heeman}

XXX what superaccepts do we play?

\chapter{The 2\c\ opening}

The 2\c\ opening shows 10--14 \hcp\ and a 6+ card club suit.  Opener may have a four or five card major, though with 6-5 and a minimum open 1 of the major.

With most non-passing hands, responder will bid 2\d.  The following are the exceptions:
\begin{enumerate}
\item Long major, 8-12: bid 2 of the major.
\item Long diamonds, no 4 card major: bid 2\nt.
\item Two suiter outside clubs (though with 5-5 you should often bid 2\d): with majors bid 3 \d, with a major and 6+ \d bid 3 of the major.
\item You know where to play already.
\end{enumerate}

These auctions often end at 3 \nt, 

\begin{figure}[ht]
\begin{tabular}{llllll}
2\c\\
&2\h &&&& 8--12, 5\+ \h, \nf.\\
&    &2\s &&& natural, 4\+ \s, bad \h, \nf.\\
&    &2\nt&&& max, 0--1 \h, \d\ and \s\ stoppers, \nf.\\
&    &    &3\c&& to play.\\
&    &    &3\d&& 5\+ \d, \nf.\\
&    &    &3\nt&& to play.\\
&    &3\c&&& natural, \nf.\\
&    &3\d&&& 5\+ \d, \nf.\\
&    &3\h&&& 3 \h, min, \nf.\\
&    &3\s&&& 5\+ \s, very good hand.  Almost forcing.\\
&    &3\nt&&& to play.\\
&    &4\c&&& good hand, 7 \c, 4\h.\\
&    &4\d&&& splinter.\\
&    &4\h&&& to play.\\
&2\s &&&& 8--12, 5\+ \s, \nf.\\
&    &2\nt&&& max, 0--1 \s, \d\ and \h\ stoppers, \nf.\\
&    &    &3\c&& to play.\\
&    &    &3\d,3\h&& 5\+ card suit, \nf.\\
&    &    &3\nt&& to play.\\
&    &3\c&&& natural, \nf.\\
&    &3\d&&& 5\+ \d, \nf.\\
&    &3\h&&& 5\+ \h, \nf.\\
&    &3\s&&& 3 \s, min, \nf.\\
&    &3\nt&&& to play.\\
&    &4\c&&& good hand, 7 \c, 4\s.\\
&    &4\d,4\h&&& splinter.\\
&    &4\s&&& to play.\\
&2\nt&&&&transfer to diamonds, no four card major.\\
&    &3\c &&& 0--2 diamonds.\\
&    &    &3\d && to play.\\
&    &    &3\h && stopper.  3\s is then a partial stopper.\\
&    &    &3\s && stopper.\\
&    &    &3\nt&&to play\\
&    &    &4\c &&invitational to 5\c\\
&    &    &4\d &&forcing, strong diamond suit.\\
&    &    &    &4\h & Redwood for diamonds.\\
&    &    &4\h && Redwood for clubs.\\
&    &    &4\s,4\nt && Major controls, sets \c\ as trump.\\
&    &    &5\c,5\d&& to play.\\
&    &3\d &&& 3 \d.\\
&    &    &3\h,3\s && stopper, \gf. 3\s\ over 3\h\ is a partial stopper.\\
&    &    &    & 3\nt & to play.\\
&    &    &    & 4\c,4\s & controls.\\
&    &    &    & 4\d &4\+\d.\\
&    &    &    & 4\h & Redwood.\\
&    &    &    & 4\nt & heart control.\\
&    &    &3\nt && to play.\\
&    &    &4\c && natural, forcing.\\
&    &    &    & 4\h  & six-keycard RKC.\\
&    &    &4\d && forcing slam try.\\
&    &    &4\h && Redwood.\\
&    &    &4\s && Control showing, slam try.\\
&    &    &\hbox to 0pt{5\c,5\d,6\c,6\d\hss} && to play.\\
&    &3\h,3\s &&& 4+\d, splinter.\\
&    &3\nt &&& to play, usually 8+ tricks in hand.\\
&    &4\c &&& 7\+ \c, 4\+ \d, very strong.\\
&    &4\d &&& Redwood for diamonds.\\
&    &4\h,4\s &&& 5 card major, pass or correct.\\
\end{tabular}
\caption{2\h, 2\s\ and 2\nt\ over a 2\c\ opener.}
\end{figure}

\begin{figure}[ht]
\begin{tabular}{llllll}
2\c \\
&3\c &&&& slam try in clubs by unpassed hand, usually not 4 card major.\\
&    & \hbox to 0pt{3\d,3\h,3\s\hss} &&& shortness.  Now bidding is controls, 4\d\ = Redwood.\\
&    & 3\nt &&& No shortness, not good for slam.  Now bidding is controls, 4\d\ = Redwood.\\
&    & 4\c &&& No shortness, but good hand.  Now bidding is controls, 4\d\ = Redwood.\\
&    & \hbox to 0pt{4\d,4\h,4\s\hss} &&& Second five card suit.  Now bidding is controls, 4\nt\ = RKC.\\
&3\d &&&& 5\+ \h, 5\+ \s, invitational or better.  With 5-5 invitational, can also bid 2\d.\\
&    & \hbox to 0pt{3\h, 3\s\hss} &&& 2 or 3 card fit, not enough for game, \nf.\\
&    &    & \hbox to 0pt{3\s\ (over 3\h)\hss} && 6 \s, \gf. \\
&    &    & 3\nt && diamond stopper, offering spot.\\
&    &    & 4\c && forcing, 2\+ \c.\\
&    &    &    & 4\d & control, toward 6\c.\\
&    &    &    & 4\h,4\s & If same as first bid major, shows 3.  Otherwise, control, toward 6\c.\\
&    &    &    & 5\c & to play.\\
&    &    & 4\d && slam try in opener's bid major.\\
&    & 3\nt &&& diamond stopper, no interest in majors.\\
&    & 4\c &&& nothing but clubs, \nf.\\
&    & \hbox to 0pt{4\d,4\h\hss} &&& transfer. Either side can continue past 4\h.  4\h - 4\s is Kickback.\\
&\hbox to 0pt{3\h, 3\s\hss} &&&& 6\+ \d, 5 in major, \gf.  By \ph: good 6 card suit, 4 in other major.\\
&    & \hbox to 0pt{3\s\ (over 3\h)\hss} &&& Opp. \uph: Heart support, slammish.  Opp. \ph: to play.\\
&    & 3\nt &&& to play.\\
&    & 4\c &&& Can't bid \nt\ or support.\\
&    & 4\d &&& Diamond support, slammish.\\
&    &    & 4\h && Redwood.\\
&    & \hbox to 0pt{4\h, 4\s\hss} &&& to play.\\
&    & \hbox to 0pt{4\h\ (over 3\s)\hss} &&& Opp. \uph: Spade support, slammish.  Opp. \ph: to play.\\
&3\nt &&&& To play.\\
&4\c &&&& Preemptive.\\
&4\d &&&& Redwood.\\
&\hbox to 0pt{4\h,4\s\hss} &&&& To play.\\
&4\nt &&&& Blackwood.\\
&5\c &&&& to play.\\
\end{tabular}
\caption{3\+ level responses over a 2\c\ opener.}
\end{figure}

\begin{figure}[ht]
\begin{tabular}{lllllll}
2\c \\
&2\d &&&&& general inquiry. \\
&   & 2\h &&&& 4 card major. \\
&   &   & 2\s &&& Asks which. \\
&   &   &   & 2\nt && Hearts, any strength. \\
&   &   &   &   & 3\c & signoff. \\
&   &   &   &   & 3\d & 4-6 or 4-7 spades and diamonds, \nf. \\
&   &   &   &   & 3\h & \inv \\
&   &   &   &   & 3\s & 5\s, \gf \\
&   &   &   &   & 3\nt & to play. \\
&   &   &   &   & 4\c & slam try in clubs (with spades and clubs). \\
&   &   &   &   & 4\d & slam try in hearts. \\
&   &   &   &   & 4\h & to play. \\
&   &   &   &   & 4\s & kickback. \\
&   &   &   & 3\c && spades, min. \\
&   &   &   &   & 3\d & 4-6 or 4-7 hearts and diamonds, \nf. \\
&   &   &   &   & 3\h & 5\h, \gf \\
&   &   &   &   & 3\s & \inv \\
&   &   &   &   & 3\nt & to play. \\
&   &   &   &   & 4\c & slam try in clubs (with hearts and clubs).  Opp. \ph inv to 5\c. \\
&   &   &   &   & 4\d,4\h & slam tries in spades, controls. \\
&   &   &   &   & 4\s & to play. \\
&   &   &   &   & 4\nt & RKC. \\
&   &   &   & 3\d && spades, max. \\
&   &   &   &   & 3\h & 5\h, \gf \\
&   &   &   &   & 3\s & sets spades, starts cuebidding. \\
&   &   &   &   & 3\nt & to play. \\
&   &   &   &   & 4\c & slam try in clubs (with hearts and clubs).  Opp. \ph inv to 5\c. \\
&   &   &   &   & 4\d,4\h & slam tries in spades, controls. \\
&   &   &   &   & 4\s & to play. \\
&   &   &   &   & 4\nt & RKC. \\
&   &   & 2\nt &&& natural invitation, no 4 card major, 5+ \d. \\
&   &   & 3\c &&& invitational in clubs. \\
&   &   & 3\d &&& 6+\d, 4 card major, \gf. \\
&   &   &   & 3M && Nat (4\nt = \inv, 4\c and 4\d forcing, 4 oM is slam try). \\
&   &   & 3M &&& 6M, not 4oM, \gf. \\
\end{tabular}
\caption{2\c-2\d-2\h.}
\end{figure}

\begin{figure}[ht]
\begin{tabular}{lllllll}
2\c & 2\d\\
&   & 2\s &&&& min 6332 or 7222; or one singleton/void but no 4\+ card side suit. \\
&   &   & 2\nt &&& \uph: high card \gf. \ph: \inv. \\
&   &   &   & 3\c && 6322 or 7222. \\
&   &   &   &   & 3\d,3M & Natural, forcing.  Looking for fit. \\
&   &   &   &   & 3\nt & to play. \\
&   &   &   &   & 4\c & slam try. \\
&   &   &   &   & 4\nt & slam try. \\
&   &   &   & 3\d && diamond shortness. \\
&   &   &   &   & 3M & Natural, forcing.  Looking for fit. \\
&   &   &   & 3\h && heart shortness. \\
&   &   &   & 3\s && shortness shortness, three hearts (4\s = RKC for hearts). \\
&   &   &   & 3\nt && shortness shortness, not three hearts (4\s = RKC for hearts). \\
&   &   & 3\c &&& \inv, willing to play in 3\c. \\
&   &   &   & 3\d && diamond shortness, max. \\
&   &   &   &   & 3M & 5 cards, forcing.  Openers bids of 4\c and above show support.\\
&   &   &   &   & 3\nt & to play.\\
&   &   &   &   & 4\c & natural, \nf.\\
&   &   &   &   & 4\d & Redwood for clubs.\\
&   &   &   &   & 5\c & to play.\\
&   &   &   & 3\h && heart shortness. Analogous continuations.\\
&   &   &   & 3\s && shortness shortness, three hearts. Analogous continuations. \\
&   &   &   & 3\nt && shortness shortness, not three hearts. Analogous continuations. \\
&   &   & 3\d &&& 6\+\d, 4 card major.  \nf \\
&   &   &   & 3M && Shortness, diamond fit.\\
&   &   & 3M &&& Strong, very long strong major. \\
&   &   & 3\nt &&& to play. \\
&   & 2\nt &&&& 4\d, any strength. \\
&   &   & 3\c &&& To play if opener is a min. \\
&   &   &   & 3\d && 1=1=4=7 and a good hand \\
&   &   &   & 3\h && 2 or 3 card fragment, good hand \\
&   &   &   &   & 3\s & looking for 3 hearts, forcing \\
&   &   &   &   & 3\nt,4\c,4\h & to play\\
&   &   &   & 3\s && 2 or 3 card fragment, good hand \\
&   &   &   &   & 3\nt,4\c,4\s & to play\\
&   &   &   &   & 4\h & offers choice of 4\s and 5\c \\
&   &   &   & 3\nt && no major fragment, plenty of tricks if responder can stop majors. \\
&   &   &   &   & 4\c,5\c & to play\\
&   &   &   &   & 4\d & Redwood for clubs\\
&   &   &   &   & 4M & controls toward slam in clubs\\
&   &   & 3\d &&& To play if opener is a min. \\
&   &   &   & 3\h && 2 or 3 card fragment, good hand \\
&   &   &   &   & 3\s & looking for 3 hearts, forcing \\
&   &   &   &   & 3\nt,4\d,4\h & to play\\
&   &   &   &   & 4\c & invitational to 5\d\\
&   &   &   & 3\s && 2 or 3 card fragment, good hand \\
&   &   &   &   & 3\nt,4\d,4\s & to play\\
&   &   &   &   & 4\c & invitational to 5\d\\
&   &   &   &   & 4\h & offers choice of 4\s and 5\d \\
&   &   &   & 3\nt && no major fragment, plenty of tricks if responder can stop majors. \\
&   &   &   &   & 4\c & invitational to 5\d\\
&   &   &   &   & 4\d,5\d & to play\\
&   &   &   &   & 4M & controls toward slam in diamonds\\
&   &   &   & 4\c && No major fragment, non-solid clubs, forcing \\
\end{tabular}
\caption{2\c-2\d-2\s and part of 2\c-2\d-2\nt.}
\end{figure}

\begin{figure}[ht]
\begin{tabular}{lllllll}
2\c & 2\d\\
&   & 2\nt\\
&   &   & 3\h &&& Natural, \gf, usually 5 hearts. \\
&   &   &   & 3\s && 2 card fragment \\

\end{tabular}
\caption{2\c-2\d-2\s and part of 2\c-2\d-2\nt.}
\end{figure}
\newpage

\chapter{The 2\d\ opening}

This opening shows a 6 card diamond suit and 10 to a bad 15 points.
However, with 6322 shape and at least a half-stopper (Qx, Jxx) in each
side suit we will usually treat the hand as balanced and open 1\d\ or
1\nt.  So a 2\d\ opening tends to be unbalanced or have a good suit,
and as such is a decent hand at minimum---something like \s x \h xxx
\d KQJxxx \c KJx.  A maximum hand for the opening is \s xx \h Ax \d
AKxxxx \c QJx.

If responder has a good fit, it doesn't take a lot of high cards to
make a game or even a slam: for instance \s x \h xxx \d xxxx \c AKxxx
makes a good slam opposite the second example hand above, and a good
5\d\ contract opposite many lesser maximum hands.  Without a fit, one
should have at least a good 10 count to respond, bearing in mind that
doing so essentially forces the auction to 3\d.

There is not a lot of room over a 2\d\ opening, so we give up on some
niceties such as ultra-scientific slam bidding and being able to play
\nt\ contracts from both sides.  We do try to decide intelligently
between 3\nt\ and 5\d, so if responder is not interested in 5\d, he
should strive to bid 3\nt\ as quickly as possible, even if it may
appear to be wrongsiding the contract.  Similarly, in slam auctions,
responder may decide not to investigate a side fit if there is already
a known good diamond fit, in order to simplify the auction.

The initial responses to 2\d\ are arranged as follows, with the raises
listed first.

\begin{figure}[ht]
\begin{tabular}{lll}
2\d&$n$\d&Preemptive raise\\
   &2\nt&Constructive or better \d\ ``raise'' (might be 2 cards).  Subsequent 4\h\ is 1430.\\
   &3\M&Splinter raise\\
   &4\c&Serious splinter raise\\
\noalign{\hrule}
   &2\M&Natural, forcing, 4\+ cards\\
   &3\c&Natural, \gf, no \d\ fit\\
   &3\nt&To play\\
   &4\M&To play
\end{tabular}
\caption{Responses to 2\d.}
\end{figure}

After the 2\nt\ raise, opener bids 3\d\ with a real minimum, 3\c\ with
an intermediate strength hand, and higher with a maximum.  This allows
the 2\nt\ raise to be a little lighter than a normal invite.  Bids of
3\M\ are showing stoppers for 3\nt.

We have some artificial methods after a 2\M\ response to find 5-3 fits
as well as sort out strength and stoppers.
\begin{itemize}
\item If opener has four-card support for the major, we always play in
  game.  The four suit bids from 3\M+1 through 4\M\ show max with high
  shortage, max with low shortage, min with high shortage, min with
  low shortage in order.
\item With three-card support, opener bids 3\c.  This is not specific
  as to strength.  With only an invitational hand, responder rebids
  3\M\ with a fit and 3\d\ without.  With a game-going hand, responder
  can sign off in 3\nt\ or 4\M, make a slam try in \d\ with 4\d, or
  make a cuebid agreeing the major with another bid.  (What is
  2\d--2\s--3\c--3\h?)
\item Without support but with four spades, opener bids 2\s.  Further
  bidding is natural; 3\s\ presumably shows a spade fit and is
  therefore forcing.  Perhaps 3\c\ is a general force.  Really, there
  is more room than we need in this rare auction.
\item Otherwise (without support or four spades) opener bids
  2\nt\ with a minimum, or something higher with a maximum.  Over
  2\nt, 3\c\ is game forcing asking for further description as though
  opener had a maximum, and 3\d\ and 3\M\ are natural and non-forcing.
  The higher descriptive rebids by opener are
\begin{itemize}
\item 3\d\ with a club stopper.  Then 3\h\ shows a six-card suit,
  asking for doubleton support, with 3\s\ denying a fit or a stopper
  in the other major, 3\nt\ showing a stopper in the other major, and
  higher bids showing a fit; while 3\s\ asks for a stopper in the
  other major, and 3\nt\ is to play.
\item 3\h\ with no club stopper and usually decent two-card support,
  but occasionally a stopper-less hand without support.  Then
  3\s\ asks for a stopper in the other major.
\item 3\s\ with four hearts (when \M\ = \s).
\item 3\nt\ with no club stopper but a stopper in the other major.
\end{itemize}
\end{itemize}

As a general rule, 4\d\ is a slam try for \d.  We do not play 4\d\ in
a constructive auction.  As usual, 4\h\ is 1430 for \d, but only once
\d\ have been agreed.

\chapter{The 2\nt\ opening}

20--21 \bal\ \fs, 21--22 \bal\ \tf. % XXX is it ever 5422? 6322?

The only reliable way for responder to get out below 3\nt\ is to pass.
Responder is allowed to make a bid with a weak hand, but may be held
responsible for the consequences.

We play ``semi-puppet'' Stayman: opener's 3\h\ rebid could be 4 or 5
but 3\s\ guarantees 5.  (Mnemonic: the higher bid is more
informative.)  Opener rebids 3\nt\ with both majors.  All suit bids
from 3\d\ to 4\s\ are transfers except 3\s\ which shows both minors.
However, opener may reject the transfers (except those to 4\M) in
specific situations.  Opener is expected to superaccept a transfer to
3\M\ with any hand with four-card support, to help responder judge in
slam auctions.

As a result of these methods, responder should always transfer with 5
hearts and shorter spades, but may bid Stayman with 5 spades and equal
or shorter hearts.  For full details, see the following table.

\begin{figure}[ht]
\begin{tabular}{lllll}
2\nt&3\c&&&Stayman; promises a 3\+-card major, not exactly 35 or 45 in the majors\\
    &   &3\d&&Denies four hearts, may have exactly four spades\\
    &   &   &3\h&Shows four spades.  Opener bids 3\s\ with a fit, 3\nt\ otherwise.\\
    &   &   &3\s&Shows five spades and presumably at least three hearts.\\
    &   &   &   &With a slam invite and 55 majors, bid 4\h\ next over opener's 3\nt.\\
    &   &   &4m&Natural\\
    &   &   &4\h&55 majors, strong slam try or better, forcing to 4\nt\\
    &   &3\h&&Shows four or five hearts, denies four spades\\
    &   &   &3\s&How many hearts?  Could be slam try with 4 (subsequent bids agree \h).\\
    &   &   &   &Opener's 3\nt\ shows 4, others 5, with new suits seminatural with good hands.\\
    &   &   &4m&Natural\\
    &   &3\s&&Shows five spades\\
    &   &   &4m&Natural\\
    &   &   &4\h&Slam try agreeing \s\\
    &   &3\nt&&Shows four cards in each major.  Responder \textbf{transfers} as over 2\nt.\\
    &3\d&&&Transfer to \h\ (always 5\+)\\
    &   &3\h&&Any normal hand: not 5233, and not four-card support.\\
    &   &   &3\s&Natural, shows four spades\\
    &   &   &4m&Natural, unbalanced and at least four cards, slam try\\
    &   &   &4\h&NF slam try\\
    &   &3\s&&Exactly 5233.  This is not a superaccept!  All continuations are natural.\\
    &   &3\nt&&Exactly 3433.  Other superaccepts show a doubleton somewhere.\\
    &   &4m&&Natural superaccept (5422 or maybe 4432 shape)\\ % XXX do we tend to have 5422s?
    &   &4\h&&Four-card support and a doubleton, but otherwise unexceptional\\
    &3\h&&&Transfer to \s\ (always 5\+)\\
    &   &3\s&&Any normal hand: not 2533, and not four-card support.\\
    &   &   &4m&Natural, unbalanced and at least four cards, slam try\\
    &   &   &4\h&55 majors, no slam interest\\
    &   &   &4\s&NF slam try\\
    &   &3\nt&&Exactly 2533.  This is not a superaccept!  All continuations are natural.\\
    &   &4\x&&Natural superaccept (5422 or maybe 4432 shape)\\ % XXX do we tend to have 5422s?
    &   &4\s&&Four-card support, unexceptional hand; could be 4333.\\
    &3\s&&&Both minors (55 or (64) or more), \gf\ but not necessarily slam invitational\\
    &   &3\nt&&Natural, wastage/stoppers in both majors, NF\\
    &   &    &4m&longer minor\\
    &   &    &4\M&short major, minors of equal length\\
    &   &4m&&Good hand for slam in the minor bid but not the other minor\\
    &   &4\M&&Good hand for slam in both suits, cuebid\\
    &4\c&&&Transfer to \d; opener can reject with 4\nt\\
    &4\d&&&Transfer to \h; then 4\s\ is 1430\\
    &4\h&&&Transfer to \s; then 4\nt\ is 1430\\
    &4\s&&&Transfer to \c; opener can reject with 4\nt
\end{tabular}
\caption{Stayman, transfers and minor suit Stayman over a 2\nt\ opening.}
\end{figure}

With exactly 53 in the majors, responder has two options, both of
which are guaranteed to find a major suit fit if one exists: either
transfer to spades, which opener will reject with a doubleton in
spades and five hearts; or bid Stayman and then bid 3\s\ over 3\d\ or
3\h.  The latter will play a \h\ fit in preference to a \s\ fit, which
might be preferable if responder has a singleton; however, spade
contracts will then be played from the ``wrong'' side.

With 55 in the majors, and
\begin{itemize}
\item no slam interest, responder should transfer to \s\ and then bid 4\h;
\item mild slam interest, responder should bid Stayman, then
  3\s\ assuming opener bids 3\d, then 4\h\ if opener bids 3\nt\ (if
  not, then opener will have expressed some opinion about a \s\ slam);
\item strong slam interest or more, responder should bid Stayman, then
  jump to 4\h\ (forcing to 4\nt).
\end{itemize}

\chapter{Appendix: Artificial Bids}
We collect a list of artificial bids here as a memory aid.  We only include bids over 1\d, 1\h, 1\s, 2\c and 2\d.

\textbf{Warning}: most of this is now out of date.

\begin{figure}[ht]
\begin{tabular}{lllllllll}
1\d & 1\h\\
& & 1\s & 2\c &&&& & Fourth suit forcing.\\
& & & & 2\h & 3\c && & asking for club stopper. \\
& & & & 3\h & 3\s && & artificial ask. \\
& & & & 3\h & 3\s & 4\c & & denies club stopper, at most four diamonds. \\
& & & & 3\h & 4\c & & & slam try in hearts. \\
& & & 2\nt & 3\c & & & & Forcing, warns of short hearts. \\
& & & 2\nt & 3\s & & & & Forcing, warns of short clubs. \\
& & & \hbox to 0pt{4\c,4\d} & & & & & Splinter. \\
& & 1\nt & 2\c & & & & & Invitational NMF (or transfer to \d). \\
& & & 2\d & & & & & GF NMF (by UPH). \\
& & & & 2\s & & & & Minimum, no heart support. \\
& & & \hbox to 0pt{4\c, 4\d, 4\s} & & & & & Splinters with big heart suit. \\
& & 2\c & 2\s & & & & & Fourth suit forcing. \\
& & & & 3\s & & & & Catchall. \\
& & & 3\s & & & & & Splinter by PH.\\
& & & 4\s & & & & & Splinter. Void by PH.\\
& & 2\d & 2\s & & & & & General purpose force by UPH, values in spades. \\
& & & 3\c & & & & & General purpose force by UPH, values in clubs. \\
& & & 3\s & & & & & Splinter by a PH. \\
& & & \hbox to 0pt{4\c, 4\s} & & & & & Splinter (4\s shows void by PH). \\ 
& & 2\h & 2\s & & & & & Art. force with 4 hearts (and 4+ card minor if inv) (table below). \\
& & & 2\nt & 3\s & & & & 1=3=6=3 (or 1=3=7=2 or 0=3=7=3), extras. \\
& & & 3\c & & & & & By UPH, 5+ hearts, relatively balanced, slam interest.\\
& & & & 3\h & & & & 1=3=(5-4), min. \\
& & & & 3\s & & & & balanced min, 3 hearts.  \\
& & & & \hbox to 0pt{4\c, 4\d} & & & & Splinter (for clubs, spades, resp), 4 hearts. \\
& & & 3\d & & & & & By UPH, shows a 6511 (or 7411) shape of some kind. \\
& & & \hbox to 0pt{3\s, 4\c, 4\d} & & & & & Splinter. \\
& & 2\s & 2\nt & & & & & Lebensohl-style puppet to 3\c. \\
& & & 3\c & & & & & Fourth suit forcing. \\
& & & \hbox to 0pt{4\c, 4\d} & & & & & Splinter. \\
& & 2\nt & & & & & & A hand worth 3\d but with 3 hearts. \\
& & & 4\c & & & & & Splinter, hearts as trump. \\
& & 3\h & 3\s & & & & & Asking for shortness. \\
& & & & 3\nt & & & & Short spades. \\
& & & & 4\c & & & & Short clubs. \\
& & & & 4\d & & & & No singleton or void. \\
& & \hbox to 0pt{3\s, 4\c} & & & & & & Singleton splinter. \\
& & 3\nt & & & & & & Void somewhere. \\

\end{tabular}
\caption{Artificial bids over 1\d-1\h.}
\end{figure}

\begin{figure}[ht]
\begin{tabular}{ll|ll|l}
 2\nt & & & & Min, 2=3=3=5, 3=3=2=5, 2=3=5=3 or 3=3=5=2, NF. \\
 & \hbox to 0pt{3\c, 3\d} & & & To play, but opener can raise with good support. \\
 & 3\h & & & Asks for five card minor. \\
 & & 3\s & & Five clubs. 3\nt, 4\nt\ natural, 4\c\ sets clubs, 4\d\ Redwood, 4\h, 4\s\ control.\\\
 & & 3\nt & & Five diamonds. 4\nt\ natural, 4\c, 4\s\ control, 4\d\ sets diamonds, 4\h\ Redwood.\\
 & 3\s & & & Asks for spade stopper. 4\c, 4\d\ show five card suit, deny stopper.\\
 & 3\nt & & & To play. \\
 3\c & & & & Min, 1=3=(5-4), 2=3=4=4, 0=3=5=5 or 0=3=6=4. \\
 & 3\d & & & To play. \\
 & 3\h & & & GF, asking which shape. \\
 & & 3\s & & 2=3=4=4, no spade stop. \\
 & & 4\h & & 0=3=5=5 or maybe 0=3=6=4. \\
 & 3\s & & & Relay to 3\nt, looking for 4 or 5 of minor. \\
 3\d & & & & Max, 3=3=6=1, 3=3=7=0 or 2=3=6=2. \\
 & 3\h & & & Asks shape. \\
 & & 3\s & & 2=3=6=2. 3\nt, 4\h, 4\nt\ natural, 4\c\ control, 4\d\ sets diamonds.\\
 & & 3\nt & & 3=3=6=1 or 3=3=7=0. 3\nt, 4\h, 4\nt\ natural, 4\c\ control, 4\d\ sets diamonds.\\
 & 3\s & & & Asks for spade help. \\
 & 3\nt & & & To play. \\
 & 4\c & & & Sets diamonds, club control. \\
 & 4\d & & & Sets diamonds, no club control.\\
 & 4\h & & & To play. \\
 3\h & & & & Balanced min, 4 hearts.\\
 3\s & & & & Max, 1=3=6=3 (or 0=3=7=3). 3\nt, 4\d, 4\h, 4\nt, 5\d\ natural, 4\c, 4\s\ control.\\
 3\nt & & & & 3 card hearts, balanced max (2=3=4=4, 2=3=(5-3), 3=3=(5-2)). \\
 & 4\c & & & 1=4=4=4, 0=4=4=5, 4=4=1=4, or 4=4=0=5 looking for club (or diamond) slam. \\
 & & 4\d & & 2=3=5=3 or 3=3=5=2. \\
 & & & 4\h & Redwood for diamonds. \\
 & & & 4\s & Spade void? \\
 & & & 4\nt & To play. \\
 & & 4\h & & Redwood for clubs. \\
 & & 4\s & & Spade control. \\
 & & & 4\nt & Last chance RKC. \\
 & & 4\nt & & Heart control? \\
 & 4\d & & & 4=4=4=1 or 4=4=5=0 looking for diamond slam.\\
 & & 4\h & & Redwood. \\
 & & 4\s & & Spade control. \\
 & & & 4\nt & Last chance RKC. \\
 & & 4\nt & & To play.\\
 4\c & & & & club splinter, four hearts. \\
 4\d & & & & spade splinter, four hearts. \\
 4\h & & & & Fourth heart, balanced or semi-balanced max. 4\s\ RKC. \\
\end{tabular}
\caption{1\d - 1\h - 2\h - 2\s\ module.}
\end{figure}

\begin{figure}[ht]
\begin{tabular}{ll|ll|ll|ll|l}
1\d & 1\s \\
& & 1\nt & 2\c & & & & & Invitational NMF (or transfer to \d). \\
& & & 2\d & & & & & GF NMF (by UPH). \\
& & & \hbox to 0pt{4\c, 4\d, 4\h} & & & & & Splinters with big spade suit. \\
& & 2\c & 2\h & & & & & Fourth suit forcing, denies inv with 4-card-minor. \\
& & & & 2\nt & & & & 1=3=(5-4), forcing: extras but no heart stopper. \\
& & & & 3\h & & & & Game forcing, heart weakness. \\
& & & & 3\nt & & & & 0=4=(5-4), any strength. \\
& & & 4\h & & & & & Splinter in support of clubs. \\
& & 2\d & 3\c & & & & & Fourth suit forcing by UPH. \\
& & & \hbox to 0pt{4\c, 4\h} & & & & & Splinter in support of diamonds. \\
& & 2\s & 2\nt & & & & & Forcing asking bid (see table below). \\
& & & \hbox to 0pt{4\c, 4\d, 4\h} & & & & & Self splinter. \\
& & 2\nt & & & & & & A hand worth 3\d but with 3 spades. \\
& & & \hbox to 0pt{4\c, 4\h} & & & & & Self splinter. \\
& & 3\s & 3\nt & & & & & Asking for shortness. \\
& & & & 4\c & & & & Short clubs. \\
& & & & 4\d & & & & No singleton or void. \\
& & & & 4\h & & & & Short hearts. \\
& & 3\nt & & & & & & Void somewhere. \\
& & \hbox to 0pt{4\c, 4\h} & & & & & & Singleton splinter. \\
& 1\nt & 2\h & & & & & & Singleton spade (or void), extras. \\
& & 2\s & & & & & & Singleton heart (or void), extras. \\
& 2\c & 2\h & & & & & & Min, not 5 \d. \\
& & & 2\s & & & & & Artificial waiting, asking about club support. \\
& & & & 3\d & & & & Awesome club fit, shortness in one major and at most 3 in other.\\
& & & & 3M & & & & 4 cards in M, 5-4 in minors. \\
& & & 3\c & \hbox to 0pt{4\c, 4\d} & & & & Great playing strength for clubs, singleton in corresponding major.\\
& & & 3\c & \hbox to 0pt{4\h, 4\s} & & & & Great playing strength for clubs, void in bid major.\\
& & 2\s & & & & & & Artificial game force, at least one 4 card major. \\
& & 3\h & 3\s & & & & & Slam try in hearts. \\
& & 3\s & 4\h & & & & & Slam try in spades. \\
& & 4\d & & & & & & Redwood. \\
& & \hbox to 0pt{4\h, 4\d} & & & & & & Splinters, very strong.\\
& 2\d & 2\h & & & & & & Minimum balanced, at least 3 diamonds. \\
& & 2\s & & & & & & Minimum with singleton or void. \\
& & 3\c & & & & & & Balanced or semibalanced max, at least 3 diamonds. \\
& & \hbox to 0pt{3\d, 3\h, 3\s} & & & & & & Shortness (3\d = clubs) in a good hand.\\
& & 4\c & 4\d & & & & & RKC. \\
& & 4\h & 4\s & & & & & RKC. \\
& & 4\s & 4\nt & & & & & RKC. \\
& 3\c & \hbox to 0pt{3\h, 3\s, 4\h, 4\s} & & & & & & Splinter (4 level shows void). \\
& \hbox to 0pt{3\h, 3\s, 4\c} & & & & & & & Splinter, six card support. \\
& 3\nt & 5\c & & & & & & Super Gerber? \\

\end{tabular}
\caption{Artificial bids over 1\d-1\s+.}
\end{figure}

\begin{thebibliography}{9}
\bibitem{Heeman}
Heeman system notes.  \url{http://www.jackbridge.com/pdf/eheeman.pdf}
\bibitem{IMprecision}
IMprecision system notes.  \url{http://www.cs.ucla.edu/~awm/bridge/IMprecision.pdf}
\bibitem{Revision}
Revision system notes.  \url{http://bridgewithdan.com/systems/revision_club_4th_ed.zip}
\end{thebibliography}

\end{document}
