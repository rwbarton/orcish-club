\documentclass{report}
\usepackage{fullpage}

\newcommand{\fs}{1st/2nd}
\newcommand{\tf}{3rd/4th}
\newcommand{\hcp}{\textsc{hcp}}
\newcommand{\bal}{\textsc{bal}}
\newcommand{\ak}{\textsc{ak}}
\newcommand{\gf}{\textsc{gf}}
\newcommand{\nf}{\textsc{nf}}
\renewcommand{\c}{\ensuremath{\clubsuit}}
\renewcommand{\d}{\ensuremath{\diamondsuit}}
\newcommand{\h}{\ensuremath{\heartsuit}}
\newcommand{\s}{\ensuremath{\spadesuit}}
\newcommand{\nt}{\textsc{nt}}
\newcommand{\p}{\textsc{pass}}
\newcommand{\x}{\textsc{x}}
\newcommand{\y}{\textsc{y}}
\newcommand{\+}{\ensuremath{^+}}
\newcommand{\M}{\textsc{m}}
\newcommand{\OM}{\textsc{om}}
\newcommand{\modifications}{\paragraph{Modifications from Revision.}}

\title{The Orcish Club}
\begin{document}
\maketitle
\tableofcontents
\listoffigures

\chapter{Opening bid structure}

In \fs, all hands with 16+ \hcp\ or 15 \hcp\ unbalanced or with a
5-card major open 1\c, \emph{except} 19--20 \bal\ hands, which open
2\nt.  The exception is to help us show balanced hands in competition,
by passing with a minimum hand and bidding cheaply in \nt\ with 21\+.
We may open 1\c\ with 13 or 14 \hcp\ hands of unusual playing
strength.  In \tf, all these ranges are 2 points higher.

We open most 10 \hcp\ hands in \fs.  The main exception is that we do
not open a balanced 10 when vulnerable.  Hands with a five-card major
open 1\h\ or 1\s, hands with six clubs open 2\c, hands with 4414,
(34)15, or 4405 shape open 2\d, and the remaining unbalanced hands
open 1\d.  Balanced hands open 1\nt\ with 10--12 when nonvulnerable,
1\d\ with 13--15; the ranges are reversed when vulnerable.

In \tf, our opening ranges are nominally all raised by 2 points.  In
3rd however it is almost impossible for responder to have an invite
opposite 12--16 without a good fit, so we may feel free to open with
10 or 11 \hcp\ and a reason to want to bid.  Our nonvulnerable
1\nt\ opening is 10--14 in 3rd and 13--14 in 4th.  The vulnerable
\tf\ 1\nt\ opening is a standard strong 15--17.

\chapter{The 1\c\ opening}

1\d\ is the negative response, normally 0--8 \hcp, but could contain a
stronger hand with fewer than 2 \ak.  1\h\+ is \gf, transfer-oriented
symmetric relay, promising 2 \ak.

\section{1\c--1\d}
1\h\ shows any 19\+ \hcp, 1\nt\ shows 16--18 \bal, and suit bids
from 1\s\ to 2\h\ are natural, 5 cards, 15--18 \hcp.  Opener uses his
discretion with (4441) shape.  2\s\+ do not exist.

\subsection{1\c--1\d--1\h}
1\s\ is a second negative showing 0--4 \hcp; 1\nt\+ is \gf,
transfer-oriented symmetric relay shifted up two steps, not promising
any \ak.

\subsubsection{1\c--1\d--1\h--1\s}
1\nt\ shows 21--23 \bal.  It's sort of silly that we end up in
1\nt\ with 21 opposite 0 but 2\nt\ with 20 opposite 0, but oh well.
Strong notrump systems are on (responder can certainly have a
game-forcing hand).

2\c\ is an artificial \gf.  Continuations are natural.

2\nt\ seems not to exist.

Suit bids from 2\d\ to 3\c\ are natural, and to the extent possible,
our agreements about auctions after 1\c--1\d--2\x\ apply here too.

\subsection{1\c--1\d--1\s}
We pretend that opener opened 1\s, except his range is now 15--18
rather than 10--14, and play our usual systems.  This is reasonable
since responder could have a game-forcing hand without a spade fit
(with 0--1 \ak).

\subsection{1\c--1\d--1\nt}
Strong notrump systems are on; again, responder could have a
game-forcing hand.

\subsection{1\c--1\d--2\x}
In general, continuations are mostly natural, except that 2\nt\ by
either hand is artificial showing a minimum hand in the context of the
auction.  It is a puppet to 3\c\ when it is responders first rebid, and
otherwise mostly forcing, but may be passed if a misfit seems likely.

With a bad hand, responder must either
\begin{itemize}
\item pass,
\item jump shift, or
\item bid 2\nt, and over the 3\c\ response, pass or correct to a suit
  below 3\x.
\end{itemize}

An exactly invitational hand with a 6\+-card suit \y\ above \x\ may
bid 2\nt\dots3\y, and an invitational hand with a ``fit'' for \x\ may
bid 2\nt\dots3\x\ (unless \x=\c; then responder must have an invite,
so opener should bid the lowest contract he's willing to play).  The
invitational raise is really just suggesting to play 3\x\ when the
invite is rejected; since we cannot play 2\nt\ this may not be a real
fit and opener should probably bid 3\nt\ even holding hearts.  That
covers the immediate 2\nt\ response by responder.  (The sequence
2\nt\dots3\nt\ is undefined.)

If responder does not make an initial jump shift or 2\nt\ response,
the auction is forcing to 2\nt\ by either player.  Bypassing 2\nt\ to
bid at the 3-level establishes a game force.  The artifical 2\nt\ is
not forcing; a minimum hand can pass but will more often pull to a
known fit or to 3\c\ to play in opener's second suit.  A maximum hand
bids above 3\x\ to establish a game force (this is why a hand which is
only invitational must start with 2\nt).

An initial 2\y\ over 2\x\ only shows a 4-card suit.  When we establish
a game force rather than bidding game directly, it tends to show doubt
over strain rather than level.  So 2\x--2\y--3\y\ will usually be a
3-card (\gf) raise.  Opener can make a minimum (3- or) 4-card raise
via 2\x--2\y--2\nt--3\c--3\y.  A jump shift to a new suit after
2\x--2\y\ is a splinter raise.  (Most of that assumes \y\ is a major.
When \x\ is \c\ and \y\ is \d, a raise to 3\d\ is probably a
suggestion to play 5\d, hence a real fit.)

\chapter{The 1\h\ and 1\s\ openings}

We open 1\h\ or 1\s\ with a 5-card major and 10--14 \hcp\ in \fs,
12--16 \hcp\ in \tf\ (though we could occasionally have fewer than
12).  We open 1\h\ or 1\s\ even with a 6-card minor if the hand is
minimum.  We normally do not open 1\nt\ with a 5-card major.

Our non-jump responses are fairly similar to standard 2/1.  1\s\ over
1\h\ is natural and forcing with 4\+\ spades, and 1\nt\ is a forcing
notrump.  However, neither of these bids promises any values.  With a
non-invitational 3-card raise, we make a delayed raise through 1\nt.
In repsonse to 1\h\ with a hand of invitational strength, exactly 4
spades, and no heart fit, we start with the forcing notrump and then
bid 2\s\ on the next round; we do not use the common treatment of this
sequence as a strong raise of opener's second suit.  We also bid a
forcing 1\nt\ over 1\s\ with 1444 shape.  A jump into a new suit on
the second round (except 1\s--1\nt--2\d--4\c, which passes 3\nt\ with
no known major suit fit) reveals this hand type; the auction
1\s--1\nt--2\s--3\nt\ is also possible.

A 2/1 bid is forcing to game and is almost always a 5-card suit: the
only common exception is (4432) shape with 2 cards in opener's major.
A rebid by opener above 2 of the opened major shows extra values.

Our raise structure is less standard.  We play a raise to 2 as an
\emph{invitational} 3-card raise, about 11--13 support points.  A jump
to 3 is a mixed raise, about 8--10 points with 4-card support; opener
should usually not bid game without extra shape.  A jump to 4 is a
two-way bid, showing either a hand with a fit and game values that
sees little chance of slam, or a hand with enough shape to want to
play in game even without values (5-card support and a singleton or
4-card support and a void).  We have two ranges of splinters, shown in
Figure \ref{fig:1M-splinters}.  This leaves the invitational 4-card
raises and slammish 3-card raises.  We handle both through the
2\nt\ gadget, described in Figure \ref{fig:1M-2NT}.

The jump shift 1\h--2\s\ is weak; jump shifts to the three level are
natural and invitational.

\begin{figure}
\begin{center}
\begin{tabular}{lllll}
1\h
&3\s &&Lower-range splinter in an undisclosed suit\\
&    &3\nt&Asks which suit.  4\c\ = \c, 4\d\ = \d, 4\h\ = \s.\\
&3\nt&&Upper-range splinter in \s\\
&4\c &&Upper-range splinter in \c\\
&4\d &&Upper-range splinter in \d\\
1\s
&3\nt&&Lower-range splinter in an undisclosed suit\\
&    &4\c&Asks which suit.  4\d\ = \d, 4\h\ = \h, 4\s\ = \c.\\
&4\c &&Upper-range splinter in \c\ or \h\\
&    &4\d&Asks which suit.  4\h\ = \h, 4\s\ = \c.\\
&4\d &&Upper-range splinter in \d\\
&\hbox to 0pt{\hss(}4\h&&To play)\\
\end{tabular}
\end{center}
\caption{\label{fig:1M-splinters}Splinter raises of a 1\h\ or 1\s\ opening.}
\end{figure}

\begin{figure}
\begin{center}
\begin{tabular}{llllllll}
1\M&2\nt
 &3\c &&&&&minimum without shortness or with a singleton \OM,\\
&&    &&&&&\quad or a maximum without shortness not interested in 3\nt,\\
&&    &&&&&\quad or any hand with a void\\
&&    &3\d &&&&Responder wants to go to game opposite any hand\\
&&    &    &&&&\quad except possibly the minimum without shortness.\\
&&    &    &3\M &&&minimum without shortness\\
&&    &    &3\OM&&&\OM\ shortness, any strength\\
&&    &    &    &3\nt&&Asking for further description.\\
&&    &    &    &    &4\c&maximum, \OM\ void\\
&&    &    &    &    &4\d&minimum, \OM\ void\\
&&    &    &    &    &4\h&minimum but game-going, \OM\ singleton.\\
&&    &    &    &    &4\s&dead minimum, \OM\ singleton.\\
&&    &    &    &    &   &\quad Combined with the previous step when \M\ = \h.\\
&&    &    &3\nt&&&minimum, \c/\d\ void\\
&&    &    &    &4\c &&Asks which minor.  4\d\ = \d, 4\h\ = \c.\\
&&    &    &4\c &&&maximum, \c\ void\\
&&    &    &4\d &&&maximum, \d\ void\\
&&    &    &4\M &&&maximum without shortness not interested in 3\nt\\
&&    &3\M &&&&Responder wants to go to game only opposite a maximum. (\nf)\\
&&    &    &&&&\quad 3\nt, 4\c, 4\d\ continuations as above; 4\M\ to play.\\
&&3\d &&&&&minimum, \d\ singleton\\
&&3\h &&&&&minimum, \c\ singleton\\
&&3\s &&&&&maximum, any singleton\\
&&    &3\nt&&&&Asking for the singleton.  4\c\ = \c, 4\d\ = \d, 4\h\ = \OM.\\\
&&3\nt&&&&&maximum without shortness interested in 3\nt\\
&&4\c &&&&&maximum, 5-card \c\ suit\\
&&4\d &&&&&maximum, 5-card \d\ suit\\
&&4\h &&&&&maximum, 5-card \OM\ suit or (74)11\\
\end{tabular}
\end{center}
\caption{\label{fig:1M-2NT}The 2\nt\ gadget.}
\end{figure}

\chapter{The weak 1\nt}

When nonvulnerable, 10--12 \bal\ in \fs, 10--14 in 3rd and 13--14 in
4th.  When vulnerable, 13--15 \bal\ in \fs.  (In \tf\ we play a strong
\nt, 15--17, with different systems.)

\section{Unpassed responder}
2\h/2\s\ to play, \dots.

\section{Passed responder}
We have 0--9 opposite at most 14 so there is no game.  Everything is
to play.

\chapter{The strong 1\nt}
1\nt\ opening showing 15--17 \bal\ in \tf\ when vulnerable, or
1\c--1\d--1\nt\ or 1\c--1\d--1\h--1\s--1\nt.

\chapter{The 2\d\ opening}

The 2\d\ opening shows 10--14 \hcp\ and ``approximately 4415'': 3415,
4315, 4405 or 4414.

Responding hands not interested in game pass or bid a new suit
cheaply.  Game bids are to play and jump shifts (including 2\d--4\d)
are natural slam tries.  3\d\ is a (rare) natural invite.  The
remaining hands start with a 2\nt\ relay ask.

Invitational hands without a 5-card major, 3 clubs, or long diamonds
are problematic.  For example, if your shape is 3352, you know your
side has no 8-card fit and you don't even have a way to reach a
guaranteed 7-card fit; yet you cannot bid 2\nt\ as a natural
invitation.  It is probably best to just bid 3\nt\ even with as few as
12 points; you don't need good odds to bid games when you might not
have been making a partscore either.  With one 4-card major, however,
you can either bid it at the 2 level as a weak invite (opener can
continue with a maximum, 4-card support, and a good 5-card club suit)
or, with a stronger invite, bid 2\nt\ and then 3 of your major over a
3\c\ response.  This usually suggests a 5-card suit, but opener will
not continue without 4-card support and a good minimum.  With 44 in
the majors, you can bid 2\nt\ and continue with 3\d\ over 3\c,
provided you are willing to play in game with a double fit in the
majors.

\modifications We have modified the 2\nt\ relay from Revision Club in a few ways.
\begin{itemize}
\item When opener has a maximum hand with (43) in the majors, he bids
  the suit above his major, so that he does not declare a contract in
  that major (with his exact shape and strength known).
\item When opener has a minimum hand, and responder makes a second
  relay by bidding 3\d, the 3\nt\ and 4\c\ rebids are swapped to keep
  the more common sequence lower.
\item In the sequence 2\d--3\s--3\nt, where opener has shown 3415
  shape and responder a slam try in spades, responder's rebid of
  4\d\ is not a \d\ cuebid because opener is already known to have
  second round control of \d.  Instead, it is a slam try showing some
  values in \c\ (Queen or better).
\end{itemize}

\begin{figure}[ht]
\begin{tabular}{llllll}
2\d
&2\h &&&&to play.  But opener might continue with a great hand for \h.\\
&    &2\s &&&4405 (\nf)\\
&    &2\nt&&&3415 (\nf)\\
&2\s &&&&to play.  But opener might continue with a great hand for \s.\\
&    &2\nt&&&4315 (\nf)\\
&    &3\c &&&4405 (\nf)\\
&2\nt&&&&invitational+ relay ask\\
&    &3\c &&&any minimum (\nf)\\
&    &    &3\d &&further relay.  Promises a 4-card major.\\
&    &    &    &3\h &3415 (\nf). 4\d\ is a slam try in \h.\\
&    &    &    &3\s &4315 (\nf). 4\d\ is a slam try in \s.\\
&    &    &    &3\nt&4414. 4\h/\s\ to play, 4\c/\d\ slam try in \h/\s, step is positive.\\
&    &    &    &4\c &4405. 4\d/\h\ transfer to \h/\s.\\
&    &    &3\h &&natural, \nf, usually 5.\\
&    &    &3\s &&natural, \nf, usually 5.\\
&    &    &3\nt&&to play\\
&    &    &4\c &&invitational to 5\c\\
&    &    &4\nt&&invitational to 6\nt\\
&    &3\d &&&\gf, 4414.  A rebid below game is a natural slam try.\\
&    &3\h &&&\gf, 4405.  4\d\ is a slam try in \h.\\
&    &3\s &&&\gf, 3415.  4\d\ is a slam try in \h.\\
&    &3\nt&&&\gf, 4315.  4\d\ is a slam try in \s.\\
&3\c &&&&to play\\
&3\d &&&&natural invite\\
&3\h &&&&slam try in \h\\
&    &3\s &&&minimum, 4315\\
&    &    &3\nt&&to play\\
&    &    &4\c &&sets \c\ as trump\\
&    &    &4\d &&further try\\
&    &    &4\h &&to play\\
&    &3\nt&&&maximum, 4315.  \nf.  Continuations as above.\\
&    &4\c &&&4414 or 3415, \c\ cuebid\\
&    &4\d &&&4405\\
&    &4\h &&&4414 or 3415, no \c\ cuebid\\
&3\s &&&&slam try in \s\\
&    &3\nt&&&3415 (\nf)\\
&    &    &4\c &&sets \c\ as trump\\
&    &    &4\d &&\c\ cuebid\\
&    &    &4\h &&\h\ cuebid\\
&    &    &4\s &&to play\\
&    &4\c &&&4414 or 4315, \c\ cuebid\\
&    &4\d &&&4405\\
&    &4\h &&&4414 or 4315, \h\ cuebid, no \c\ cuebid\\
&    &4\s &&&4414 or 4315, nothing to cuebid\\
&3\nt&&&&to play\\
&4\c &&&&slam try in \c.  Usual principles; 4\d\ shows a diamond void.\\
&4\d &&&&slam try in \d.  Usual principles, except:\\
&    &4\h &&&shows a (singleton) diamond honor (Jack or better)\\
&\hbox to 0pt{4\h, 4\s, 5\c, 5\d\hss}&&&&to play
\end{tabular}
\caption{Auctions after a 2\d\ opener.}
\end{figure}

\chapter{The 2\nt\ opening}

19--20 \bal\ \fs, 21--22 \bal\ \tf.

\end{document}
